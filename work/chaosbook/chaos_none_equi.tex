\chapter{Chaos System with No Equilibrium}
\label{chap:chaosnoneequi}
\chapterauthor{
  Zhaochao Wei\\
  School of Mathematics and Statistics, South-Central University for Nationalities, Wuhan 430074, PR China\\
  \email{weizhouchao@yahoo.cn}\\
  Sajad Jafari\\
  Biomedical Engineering Department, Amirkabir University of Technology, Tehran 15875-4413, Iran\\
  \email{sajadjafari@aut.ac.ir}\\
  J.C. Sprottb\\
  S. Mohammad Reza Hashemi Golpayegani\\
  Biomedical Engineering Department, Amirkabir University of Technology, Tehran 15875-4413, Iran
}

\section{Introduction}

In this chapter, we'll introduce rare examples of simple chaotic flows with no
equilibrium. Such examples are of great interest recently due to litter
knowledge of the characteristics of them. Due to non-existence of
heteroclinic nor herteroclinic orbits, it's inapplicable for \v{S}i'lnikov theory.

In Table \ref{tab:none_tbl1}, $NE_1$ as Sprott A system \cite{Sprott1994Some} is the
oldest and best-known example for chaotic flows. Due to the belonging to 
Nose–Hoover oscillator \cite{Hoover1995Remark} which describes
many natural phenomena \cite{Posch1986Canonical}, Sprott A system suggests that
such systems may have practical as well as theoretical importance. Recently,
two other dissipative examples have been reported, which
we call the Wei system \cite{Wei2011Dynamical} listed as $NE_2$ in Table \ref{tab:none_tbl1} and the Wang–
Chen system \cite{Wang2012Constructing}, a simplified version of which with one fewer
term than previously published is listed as $NE_3$ in Table \ref{tab:none_tbl1}.
 
\section{General Method to Search Simple Chaos with Only One or No Equilibrium}
\label{sec:none_gen_met}

As a start, consider the general three-dimensional ODE's with quadratic nonlinearities of the form 
$\dot{\bm{x}}=\bm{a}+\sum\limits^{3}_{i=1}\bm{b}_i\bm{x}_i+\sum\limits^{3}_{i=1}\sum^{3}_{j=1}\bm{c}_{i,j}\bm{x}_i\bm{x}_j$,
in which $\bm{x}={x,y,z}$ is a real three-dimensional state-space variable,
and $\bm{a},\bm{b}$ and $\bm{c}$ are real three-dimensional coefficient vectors.
numerical procedure was to search hyperplanes of six or fewer dimensions in
the 30-dimensional control space of coefficients for bounded
chaotic solutions as evidenced by a positive Lyapunov exponent.

Runge-Kutta integrator method is imported to do the numerical calculation. With suitable ranges for steps for each coefficients,
altering ranges, increments, times scales and we obtain simple chaotic cases.

\section{Wang-Chen System}

First, a 3D chaotic autonomous system is introduced:

When \(a>0\), this system has two symmetrical equlibria: \((\sqrt{a},0,0)\) and \((-\sqrt{a},0,0)\), as showed
in Fig. \ref{fig:none_fig3}. When \(a=0\), these two symmetrical equlibria merge into one, the origin
\((0,0,0)\). When \(a<0\), there is no equlibrium in this system, but still, the system can generate a chaotic
attractor, as shown in Fig. \ref{fig:none_fig4}. The largest Lyapunov exponent with respect to the 
parameter a is showed in Fig. \ref{fig:none_fig5}, which convincingly implies that the system is chaotic.


To obtain Wang-Chen System, firstly, introduce a 3D chaotic autonomous system,

\begin{equation}
\label{eq:none_eq2}
    \begin{array}{l}
      \dot{x}=y\\
      \dot{y}=z \\
      \dot{z}=-y+3y^{2}-x^{2}-xz-a.
    \end{array}
  \right.
\end{equation}

By solving the equation, the equilibriums of such system are  \((\sqrt{a},0,0)\) and \((-\sqrt{a},0,0)\)
if $a>0$ as showed in Fig. \ref{fig:none_fig3} while the two equilibriums merged into one if $a=0$.
If $a<0$, no equilibrium for such a system. Nevertheless, chaos exists. As showed in Fig. \ref{fig:none_fig5}
with respect to the parameter $a$, the largest Lyapunov exponent implies the existence of
chaos convincingly.

\begin{figure}[htbp]
\centering
\includegraphics[width=0.6\textwidth]{chaos/0p-001.jpg}
\caption{\label{fig:none_fig3}
3D views of the chaotic attractor of system (\ref{eq:none_eq2}) when \(a=0.01\)}
\end{figure}

\begin{figure}[htbp]
\centering
\includegraphics[width=0.6\textwidth]{chaos/0p005.jpg}
\caption{\label{fig:none_fig4}
3D views of the chaotic attractor of system (\ref{eq:none_eq2}) when \(a=0\)}
\end{figure}

\begin{figure}[htbp]
\centering
\includegraphics[width=0.6\textwidth]{chaos/1lle.jpg}
\caption{\label{fig:none_fig5}
The largest Lyapunov exponent of system}
\end{figure}

\section{Wei System}

Just as what we did for Sprott E, a small perturbation on Sprott D may change
the type and number of equilibrium.
Sprott D system can be written as \cite{Sprott1994Some},

\begin{equation}
\label{eq:orga65b5dd}
    \begin{array}
      \dot{x} = -y\\
      \dot{y} = x+z \\
      \dot{z} = 3y^2+xz.
    \end{array}
  \right.
\end{equation}

After application of a small perturbation, the new form of system then be like,

\begin{equation}
\label{eq:none_eq1}
    \begin{array}
      \dot{x} = -y\\
      \dot{y} = cx+z \\
      \dot{z} = ay^2+xz-d.
    \end{array}
  \right.
\end{equation}

in which $a, c, d$ are real parameters. Obviously under the condition \(a = 3, c = 1, d = 0\), it's the
original Sprott D system.
When setting  \((a,c,d) = (2,1,0.35)\) and initial 
value \((−1.6,0.82,1.9)\), system (\ref{eq:none_eq1}) displays a single-scroll
chaotic attractor
with no equilibria, as shown in Fig. \ref{fig:none_fig1}(a). 
Lyapunov exponents are calculated as

\begin{equation}
  \begin{array}
    \lambda_{L_1} = 0.0793\\
    \lambda_{L_2} = 0\\
    \lambda_{L_3} = −1.5034.
  \end{array}
\end{equation}

\begin{figure}[htbp]
\centering
\includegraphics[width=0.6\textwidth]{chaos/weisys_001.png}
\caption{\label{fig:none_fig1}
Parameters \((a,c,d)=(2,1,0.35)\) and initial value \((−1.6,0.82,1.9)\) : (a) chaotic attractor with no equilibria in system \ref{eq:none_eq1}; (b) Poincare image projected in the x–z plane.}
\end{figure}

\begin{figure}[htbp]
\centering
\includegraphics[width=0.6\textwidth]{chaos/weisys_002.png}
\caption{\label{fig:none_fig2}
Three types of chaotic attractors of system \ref{eq:none_eq1} with parameters values \((a,c,d)\): (a)(2.55,1,−0.8); (b) (2.55,1,0) ; (c) (2.55,1,0.1) .}
\end{figure}

\section{Systemic Search for Systems with No Equilibrium}

Inspired by methods mentioned before, a systemic search may be applied for
three-dimensional chaotic systems with quadratic nonlinearities by adding
a constant term to each non-hyperbolic (the equilibriums have eigenvalues with a real
part equal to zero) systems. It's quite general requirement that chaotic systems
of this type includes such a constant term since there would otherwise be at
least one equilibrium point at $(0,0,0)$. Thus, three basic methods are used
for chaotic systems with no equilibrium:

\begin{enumerate}
\item The first method is adding a constant term a to other nonhyperbolic systems, e.g.

  \begin{equation}
    \left\{
    \begin{array}{l}
      \dot{x} = y \\
      \dot{y} = -x+z \\
      \dot{z} = k_1x^2+k_2z^2+k_3y^2+a 
    \end{array}
    \right.
  \end{equation}

  where \(a=0\) has an equilibrium at \((0,0,0)\) whose eigenvalues are zero and
  pure imaginary. Adjusting and simplifying the parameters
  \(k_1, k_2, k_3\), and the term $a$ gives the chaotic system listed
  as \(NE_7\) system in Table \ref{tab:none_tbl1}.

\item Second method is by looking at cases where we could show algebraically that the equilibrium
  points are imaginary. For example, any chaotic solution for a
  parametric system such as

\begin{equation}
  \left\{
    \begin{array}{l}
      \dot{x} = y \\
      \dot{y} = z \\
      \dot{z} = k_1x+k_2y+k_3z+k_4x^2+k_5y^2+k_6z^2+k_7xy+k_8xz+k_9yz+a
    \end{array}
  \right.
\end{equation}

where \(k_1^2 - 4k_4a \leq 0\) is a candicate. Adjusting the parameters \(k_1,\ldots,k_9\) and a gives the system listed as \(NE_{14}\)
in Table \ref{tab:none_tbl1}.

\item The last one is by adding a constant to each of the derivatives in known chaotic systems
  and look for solutions where the 
  numerically calculated equilibrium do not exist. For example, Case O of reference \cite{Sprott1994Some} with added constants \(a_1, a_2\) and \(a_3\),

\begin{equation}
  \left\{
    \begin{array}{l}
      \dot{x} = k_1y+a_1\\
      \dot{y} = k_2x+k_3z+a_2 \\
      \dot{z} = k_4x+k_5xz+k_6y+a_3
    \end{array}
  \right.
\end{equation}

gives the system listed as \(NE_4\) in Table \ref{tab:none_tbl1}.
\end{enumerate}

In addition to the seventeen cases listed in Tabel \ref{tab:none_tbl1}, dozens of
additional cases were found but are extensions of these
cases with additional terms. For each case been found, the space of coefficients was searched
for values that are deemed "elegant", which means
as many coefficents as possible are set to zero with the others set to \(\pm{}1\) if possible
or otherwise to a small integer or decimal fraction
with the fewest possible digits.

\begin{longtable}{|l|l|l|l|l|l|}
  \centering
\caption[Feasible triples for a highly variable Grid]{\label{tab:none_tbl1}
Seventeen Simple Chaotic with No Equilibria.} \\

\hline \multicolumn{1}{|c|}{\textbf{Model}} &
\multicolumn{1}{c|}{\textbf{Equations}} & \multicolumn{1}{c|}{\textbf{a}} &
\multicolumn{1}{c|}{\textbf{$LE_s$}} & \multicolumn{1}{c|}{\textbf{$D_{KY}$}} & \multicolumn{1}{c|}{\textbf{$(x_0,y_0,z_0)$}}\\ \hline 
\endfirsthead

\multicolumn{6}{c}%
{{\bfseries \tablename\ \thetable{} -- continued from previous page}} \\
\hline \multicolumn{1}{c|}{\textbf{Equations}} & \multicolumn{1}{c|}{\textbf{a}} &
\multicolumn{1}{c|}{\textbf{$LE_s$}} & \multicolumn{1}{c|}{\textbf{$D_{KY}$}} & \multicolumn{1}{c|}{\textbf{$(x_0,y_0,z_0)$}}\\ \hline 
\endhead

\hline \multicolumn{6}{|r|}{{Continued on next page}} \\ \hline
\endfoot

\hline \hline
\endlastfoot
$NE_1$ & $\dot{x} = y$ & 1.0 & 0.0138,0,−0.0138 & 3.0000 & (0,5,0)\\
 & $\dot{y} = −x − zy$ &  &  &  & \\
 & $\dot{z} = y^2 − a$ &  &  &  & \\
\hline
\(NE_2\) Wei & \(\dot{x} = − y\) & 0.35 & 0.0776,0,−1.5008 & 2.0517 & (0,0.4,1)\\
 & \(\dot{y} = x + z\) &  &  &  & \\
 & \(\dot{z} = 2y^2 + xz − a\) &  &  &  & \\
\hline
\(NE_3\) Simplified Wang–Chen & \(\dot{x} = y\) & 1.0 & 0.0522,0,−2.6585 & 2.0196 & (1,1,− 1)\\
 & \(\dot{y} = z\) &  &  &  & \\
 & \(\dot{z} = − y + 0.1x_2 + 1.1xz + a\) &  &  &  & \\
\hline
\(NE_4\) & \(\dot{x} = − 0 . 1 y + a\) & 1.0 & 0.0235,0,−8.480 & 2.0028 & (−8.2,0,−5)\\
 & \(\dot{y} = x + z\) &  &  &  & \\
 & \(\dot{z} = xz − 3 y\) &  &  &  & \\
\hline
\(NE_5\) & \(\dot{x} = 2y\) & 2.0 & 0.0168,0,−0.3622 & 2.0465 & (0.98,1.8,−0.7)\\
 & \(\dot{y} = − 2x − z\) &  &  &  & \\
 & \(\dot{z} = − y^2 + z^2 + a\) &  &  &  & \\
\hline
\(NE_6\) & \(\dot{x} = y\) & 0.75 & 0.0280,0,−3.4341 & 2.0082 & (0,3,−0.1)\\
 & \(\dot{y} = z\) &  &  &  & \\
 & \(\dot{z} = −y − xz − yz − a\) &  &  &  & \\
\hline
\(NE_7\) & \(\dot{x} = y\) & 2.0 & 0.0252,0,−6.8524 & 2.0037 & (0,2.3,0)\\
 & \(\dot{y} = − x + z\) &  &  &  & \\
 & \(\dot{z} = − 0.8x^2 + z^2 + a\) &  &  &  & \\
\hline
\(NE_8\) & \(\dot{x} = y\) & 1.3 & 0.0314,0,−10.2108 & 2.0031 & (0,0.1,0)\\
 & \(\dot{y} = − x − yz\) &  &  &  & \\
 & \(\dot{z} = xy + 0.5x^2 − a\) &  &  &  & \\
\hline
\(NE_9\) & \(\dot{x} = y\) & 0.55 & 0.0504,0,−0.3264 & 2.1544 & (0.5,0,0)\\
 & \(\dot{y} = −x − yz\) &  &  &  & \\
 & \(\dot{z} = −xz + 7x^2 − a\) &  &  &  & \\
\hline
\(NE_{10}\) & \(\dot{x} = z\) & 0.6 & 0.0061,0,−1.3002 & 2.0047 & (1,0.7,0.8)\\
 & \(\dot{y} = z − y\) &  &  &  & \\
 & \(\dot{z} = − 0.9y − xy + xz + a\) &  &  &  & \\
\hline
\(NE_{11}\) & \(\dot{x} = y\) & 1.0 & 0.0706,0,−0.6456 & 2.1094 & (0,1.6,3 )\\
 & \(\dot{y} = − x + z\) &  &  &  & \\
 & \(\dot{z} = z − 2xy − 1 . 8xz − a\) &  &  &  & \\
\hline
\(NE_{12}\) & \(\dot{x} = z\) & 0.1 & 0.0654,0,−2.0398 & 2.0321 & (0.5,0,−1)\\
 & \(\dot{y} = x − y\) &  &  &  & \\
 & \(\dot{z} = − 4x 2 + 8xy + yz + a\) &  &  &  & \\
\hline
\(NE_{13}\) & \(\dot{x} = − y\) & 0.4 & 0.1028,0,−2.1282 & 2.0483 & (2.5,0,0)\\
 & \(\dot{y} = x + z\) &  &  &  & \\
 & \(\dot{z} = xy + xz + 0 . 2 yz − a\) &  &  &  & \\
\hline
\(NE_{14}\) & \(\dot{x} = y\) & 1.0 & 0.0532,0,−11.8580 & 2.0045 & (1,0,−4)\\
 & \(\dot{y} = z\) &  &  &  & \\
 & \(\dot{z} = x^2 − y^2 + 2xz + yz + a\) &  &  &  & \\
\hline
\(NE_{15}\) & \(\dot{x} = y\) & 1.0 & 0.1101,0,−1.3879 & 2.0793 & (0,1,−4.9)\\
 & \(\dot{y} = z\) &  &  &  & \\
 & \(\dot{z} = x^2 − y^2 + xy + 0.4xz + a\) &  &  &  & \\
\hline
\(NE_{16}\) & \(\dot{x} = −0.8x − 0.5y^2 + xz + a\) & 1.0 & 0.0607,0,−0.1883 & 2.3224 & (0,1,−1)\\
 & \(\dot{y} = −0.8y − 0.5z^2 + yx + a\) &  &  &  & \\
 & \(\dot{z} = −0.8z − 0.5x^2 + zy + a\) &  &  &  & \\
\hline
\(NE_{17}\) & \(\dot{x} = −y − z^2 + 2.3xy + a\) & 2.0 & 0.2257,0,−1.7477 & 2.1292 & (1,−1,0)\\
 & \(\dot{y} = −z − x^2 + 2.3yz + a\) &  &  &  & \\
 & \(\dot{z} = −x − y^2 + 2.3zx + a\) &  &  &  & 
\hline
\end{longtable}

\bibliographystyle{spphys}
\bibliography{chaos_none_equi}
