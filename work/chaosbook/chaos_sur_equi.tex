\chapter{Simple Chaotic 3D Flows with Surfaces of Equilibria}
\label{chap:sur_equi}
\begin{pauthor}
  Sajad Jafari\\
  Biomedical Engineering Department,\\
  Amirkabir University of Technology,\\
  Tehran 15875-4413, Iran\\
  \email{sajadjafari@aut.ac.ir}\\
  J. C. Sprott\\
  Department of Physics, \\
  University of Wisconsin, \\
  Madison, WI 53706, USA \\
  Malihe Molaie Biomedical Engineering Department, \\
  Amirkabir University of Technology, \\
  Tehran 15875-4413, Iran
\end{pauthor}

\abstract{
  Although some four-dimensional chaotic flows with surfaces of equilibria
  have been reported, simple three-dimensional chaotic flows have not been
  discovered before. In this chapter, with a systematic computer search,
  twelve simple three-dimensional chaotic flows have been found.
  Such systems may have practical benefits for engineering.
}

\section{Introduction}
\label{sec:org110e565}

As mentioned before, simple systems of nonlinear differential
equations can generate chaos mathematically.
And now with the development of computer science, it's now possible to search
the entire space of parameters that results in desirable characteristics
of chaotic flows \cite{01_Sprott2010Elegant}.

Also, recent research has involved categorizing periodic
and chaotic attractors as either self-excited or hidden
\cite{02_Leonov2011Localization,03_Leonov2012Hidden,04_Dudkowski2016Hidden,05_Leonov2014Hidden}.
A self-excited attractor has a basin of attraction
that is associated with an unstable equilibrium,whereas
a hidden attractor has a basin of attraction that does
not intersect with vicinity of any equilibrium
points. The classical attractors of Lorenz, R\"{o}ssler,
Chua, Chen, Sprott systems (cases B to S) and other
widely known attractors are those excited from unstable
equilibria. From a computational point of view,
this allows one to use a numerical method in which
a trajectory started from a point on the unstable manifold
in the neighborhood of an unstable equilibrium,
reaches an attractor and identifies it\cite{02_Leonov2011Localization}. 
Hidden attractors cannot be found by this method and are important
in engineering applications because they allow unexpected
and potentially disastrous responses to perturbations
in a structure such as a bridge or an airplane
wing\cite{06_Leonov2015Hidden,07_Leonov2015Homoclinic,08_Sharma2015Control,09_Pooja2015Controlling,10_Dudkowski2016Hidden}.

The chaotic attractors in dynamical systems without
any equilibrium points\cite{11_Jafari2013Elementary,12_Sajad2014A,13_VietThanh2014Constructing,14_Fadhil2015A,15_Jafari2016Multiscroll,16_Pham2016A,17_Pham2016A,18_Wei2011Dynamical,19_Wang2012Constructing}, 
with only stable equilibria \cite{20_MALIHE2013SIMPLE,21_Kingni2014Three,22_Seng2014Cost,23_Pham2014Generating},
or with curves of equilibria\cite{24_Jafari2015Erratum,25_Kingni2016Three,26_Pham2016A,27_Pham2016A,28_Gotthans2015New}
are hidden attractors. That is the reason such
systems are rarely found, and only recently such examples
have been reported in the literature \cite{29_VietThanh2014Is,30_Pham2014Hidden,31_Jafari2015Recent,32_Pham2015Hidden,33_Shahzad2015Synchronization,34_Sprott2015A,35_Pham2016A,36_Pham2016A,37_Pham2016A}.
There is a similar definition for hidden attractors in chaotic
maps\cite{38_Jafari2016The,39_Jiang2016Hidden}. Even more specific systems such as
chaotic systems with multi-scroll attractors \cite{40_Li2016The,41_Ma2014Selection} can
have hidden attractors\cite{15_Jafari2016Multiscroll}.

In this chapter,we introduce a new category of chaotic
systems with hidden attractors: systems with surfaces
of equilibria. Although in such systems the basin of
attractionmay intersect the equilibrium surface in some
sections, there are usually uncountably many points on
the surface that lie outside the basin of attraction of the
chaotic attractor, and thus, it is impossible to identify
the chaotic attractor for sure by choosing an arbitrary
initial condition in the vicinity of the unstable equilibria.
In other words, from a computational point of view
these attractors are hidden. On the other hand, to the
best of our knowledge, there are no 3D chaotic systems
with surfaces of equilibria in the literature (there are
papers on 4D chaotic systems with a plane of equilibria
\cite{42_Jafari2016A}). As described in the next sections, such systems
are not difficult to construct, but they can have some
practical benefits.

\begin{table}[htbp]
  \centering
\caption{\label{tab:sur_tab1}Twelve simple chaotic flows with surface equilibrium}
\centering
\begin{tabular}{llllllrr}
\hline
\hline
Case & Surface type & Equations & \((a, b, c)\) & Equilibria & LEs & \(D_{KY}\) & \((x0, y0, z0)\)\\
\hline
\(ES_1\) & One plane & \(\dot{x} = f\times{}y\) & \(a = 1.54\) & \((0, y, z)\) & 0 & 2.0065 & 6\\
 &  & $\dot{y} =  f\times{}z$ &  &  & −1.0869 &  & 0\\
 &  & \(\dot{z} = f\times(x + ay^2 − xz)\) &  &  &  &  & −1\\
 &  & \(f = x\) &  &  &  &  & \\
\hline
\(ES_2\) & One plane & \(\dot{x} = f\times{}y\) & \(a = 1\) & \((0, y, z)\) & 0.0644 & 2.0778 & 0.15\\
 &  & \(\dot{y} = -x + az\) & \(b =3\) &  &  & 0 & 0\\
 &  & \(\dot{z} = f\times{}(by^2 − xz)\) &  &  & −0.8279 & 0.8 & \\
 &  & \(f = x\) &  &  &  &  & \\
\hline
\(ES_3\) & One plane & \(\dot{x} = f\times{}(y^2 + axy)\) & \(a = 2\) & \((0, y, z)\) & 0.0661 & 2.0397 & 0.87\\
 &  & \(\dot{y} = f\times{}(−z)\) & \(b = 1\) & \((\sqrt{\frac{b}{a}}, -\sqrt{ab}, 0)\) & 0 &  & 0.4\\
 &  & \(\dot{z} = f\times{}(b + xy)\) &  & \((-\sqrt{\frac{b}{a}}, \sqrt{ab},0)\) & −1.664 &  & 0\\
 &  & \(f=x\) &  &  & 4 &  & \\
\hline
\(ES_4\) & One plane & \(\dot{x} = f\times{}(−y)\) & \(a = 2\) & \((x, y, 0)\) & 0.0560 & 2.0516 & 0\\
 &  & \(\dot{y} = f\times(x + z)\) & \(b = 0.35\) &  & 0 &  & 0.46\\
 &  & \(\dot{z} = f\times(ay^2 + xz − b)\) &  &  & −1.0855 &  & 0.7\\
 &  & \(f = z\) &  &  &  &  & \\
\hline
\(ES_5\) & Two planes & \(\dot{x} = f\times(−az)\) & \(a = 0.4\) & \((0, y, z)\) & 0.1242 & 2.0677 & 1\\
 &  & \(\dot{y} = f\times(b + z^2 − xy)\) & \(b = 1\) & \((x, 0, z)\) & 0 &  & 1.44\\
 &  & \(\dot{z} = f\times(x^2 − xy)\) &  & \(\sqrt{b}, \sqrt{b}, 0)\) & −1.8356 &  & 0\\
 &  & \(f = xy\) &  &  &  &  & \\
\hline
\(ES_6\) & Three planes & \(\dot{x} = f\times(y + ayz)\) & \(a = 2\) & \((0, y, z)\) & 0.0294 & 2.0725 & 1\\
 &  & \(\dot{y} = f\times(bz + y^2 + cz^2)\) & \(b = 8\) & \((x, 0, z)\) & 0 &  & -1.3\\
 &  & \(\dot{z} = f\times(x^2 − y^2)\) & \(c = 7\) & \((x, y, 0)\) & −0.4051 &  & −1\\
 &  & \(f = xyz\) &  & \((\pm{}1.5, 1.5, −0.5)\) &  &  & \\
 &  &  &  & \((\pm{}1.5, −1.5, −0.5)\) &  &  & \\
\hline
\(ES_7\) & Sphere & \(\dot{x} = f\times(ay)\) & \(a = 0.4\) & \(x^2 + y^2 + z^2  = 1\) & 0.0113 & 2.0119 & 0\\
 &  & \(\dot{y} = f\times(xz)\) & \(b = 6\) & \((0, 0, 0)\) & 0 &  & 0.1\\
 &  & \(\dot{z} = f\times(-z − x^2 − byz)\) &  &  & −0.9501 &  & 0\\
 &  & \(f = 1 − x^2 − y^2 − z^2\) &  &  &  &  & \\
\hline
\(ES_8\) & Sphere & \(\dot{x} = f\times(az + y2)\) & \(a = 1\) & \(x^2 + y^2 + z^2  = 1\) & 0.0323 & 2.0338 & 0.24\\
 &  & \(\dot{y} = f\times(y + bx^2)\) & \(b = 5\) & \((0, 0, 0)\) & 0 &  & 0.2\\
 &  & \(\dot{z} = f\times(−xy)\) &  &  & −0.9552 &  & 0\\
 &  & \(f = 1 − x^2 − y^2 − z^2\) &  &  &  &  & \\
\hline
\(ES_9\) & Circular cylinder & \(\dot{x} = f\times(y^2 − axy)\) & \(a = 5\) & \(x^2 + y^2 = 1\) & 0.0388 & 2.0321 & 0.06\\
 &  & \(\dot{y} = f\times(xz)\) & \(b = 7\) & \((−0.0756,−0.3781, 0)\) & 0 &  & 0\\
 &  & \(\dot{z} = f\times(1 − by^2)\) &  & \((+0.0756,+0.3781, 0)\) & −1.2078 &  & 1\\
 &  & \(f = 1 − x^2 − y^2\) &  &  &  &  & \\
\hline
$ES_{10}$ & Hyperbolic cylinder & \(\dot{x} = f\times(a − z2)\) & \(a = 0.1\) & \(y^2 − x^2 = 1\) & 0.0420 & 2.1883 & 0\\
 &  & \(\dot{y} = f\times(xz)\) & \(b = 1\) & \((0, 0,−0.3162)\) & 0 &  & −0.08\\
 &  & \(\dot{z} = f\times(y + bxz)\) &  & \((0, 0,+0.3162)\) & −0.2230 &  & 0\\
 &  & \(f = 1 + x^2 − y^2\) &  &  &  &  & \\
\hline
\(ES_{11}\) & Paraboloid & \(\dot{x} = f\times(yz)\) & \(a = 1\) & \(x^2 + y^2 = −z\) & 0.0283 & 2.0458 & 0.46\\
 &  & \(\dot{y} = f\times(x − axz)\) & \(b = 0.6\) & \((0, y, 0)\) & 0 &  & 0\\
 &  & \(\dot{z} = f\times(x − bz^2)\) &  & \((0.6, 0, 1)\) & −0.6171 &  & 0.8\\
 &  & \(f = z + x^2 + y^2\) &  &  &  &  & \\
\hline
\(ES_{12}\) & Saddle & \(\dot{x} = f\times(yz)\) & \(a = 0.1\) & \(y^2 − x^2 = z\) & 0.0068 & 2.0135 & 1\\
 &  & \(\dot{y} = f\times(−ax)\) & \(b = 6\) & \((0, 0, 0)\) & 0 &  & 0\\
 &  & \(\dot{z} = f\times(-z + by^2 + xz)\) &  &  & −0.4998 &  & 1\\
 &  & \(f = z + x^2 − y^2\) &  &  &  &  & \\
\hline
\hline
\end{tabular}
\end{table}

\section{Simple chaotic flows with surfaces of equilibria}
\label{sec:org05d73ac}

In the search for chaotic flows with surfaces of equilibria,
we followed a simple procedure. Consider the general
parametric form of quadratic three-dimensional
flows:

\begin{equation}
\label{eq:sur_eq1}
  \left\{
  \begin{array}{l}
  \dot{x} = Q1(x, y, z)\\
  \dot{y} = Q2(x, y, z)\\
  \dot{z} = Q3(x, y, z)
  \end{array}
  \right.
\end{equation}

in which

\begin{eqnarray}
\label{eq:sur_eq2}
  Q_1 &=& a_1x + a_2y+a_3z+a_4x^2 + a_5y^2 + a_6z^2 + a_7xy\\\nonumber
  &&+ a_8xz + a_9yz + a_{10}\\\nonumber
  Q_2 &=& a_{11}x + a_{12}y + a_{13}z + a_{14}x^2 + a_{15}y^2 + a_{16}z^2\\\nonumber
  &&+ a_{17}xy + a_{18}xz + a_{19}yz + a_{20}\\\nonumber
  Q_3 &=& a_{21}x + a_{22}y + a_{23}z + a_{24}x^2 + a_{25}y^2 + a_{26}z^2\\\nonumber 
  && + a_{27}xy + a_{28}xz + a_{29}yz + a_{30}
\end{eqnarray}

In order to have a surface on which all the points are an
equilibrium, there should be a multiplying factor such
as \(f(x, y, z)\) in all the equations, so that an equilibrium
surface occurs whenever \(f(x, y, z) = 0\). Thus, the
equations to be examined are

\begin{equation}
\label{eq:sur_eq3}
  \left\{
  \begin{array}{l}
    \dot{x} = f(x, y, z) Q1\\
    \dot{y} = f(x, y, z) Q2\\
    \dot{z} = f(x, y, z) Q3
  \end{array}
  \right.
\end{equation}

The simplest candidates for the surface \(f(x, y, z)\) are
simple planes (one plane such as \(f = x\), two orthogonal
planes such as \(f = xy\) and three orthogonal
planes such as \(f = xyz\)). Also standard quadrics
(ellipsoids, hyperboloids and paraboloids) are proper
candidates.

An exhaustive computer search was done, seeking
elegant\cite{01_Sprott2010Elegant}  
dissipative cases for which the largest Lyapunov
exponent is greater than 0.001. Cases \(ES_1\sim{}ES_{12}\)
in Table \ref{tab:sur_tab1} are twelve of the simplest cases found in this
way. All these cases are dissipative with attractors projected
onto the xy plane as shown in Fig. \ref{fig:sur_fig1}. Thevalue of
parameters, Lyapunov exponent spectra and Kaplan–Yorke 
dimensions are shown in Table \ref{tab:sur_tab1} along with
initial conditions that are close to the attractor. As is
common for strange attractors from three-dimensional
autonomous systems, the attractor dimension is only
slightly greater than 2.0, the largest of which is ES10
with \(D_KY = 2.1883\), although no effort was made
to tune the parameters for maximum chaos. All the
cases appear to approach chaos through a succession
of period-doubling limit cycles, a typical example of
which (\(ES_1\)) is shown in Fig. \ref{fig:sur_fig2}. with increasing a. As
a increases further, the strange attractor is destroyed in
a boundary crisis.

\begin{figure}[htbp]
\centering
\includegraphics[width=10cm]{chaos/sur_equi_01.png}
\caption{\label{fig:sur_fig1}
State space plots of the cases in Table 1 projected onto the xy plane.}
\end{figure}

\begin{figure}[htbp]
\centering
\includegraphics[width=0.6\textwidth]{chaos/sur_equi_02.png}
\caption{\label{fig:sur_fig2}
Largest Lyapunov exponent and bifurcation diagram of case $ES_1$
showing a period-doubling route to chaos.}
\end{figure}

\begin{figure}[htbp]
\centering
\includegraphics[width=0.6\textwidth]{chaos/sur_equi_03.png}
\caption{\label{fig:sur_fig3}
  Cross section of the basins of attraction of the
  two attractors in the xz plane at y = 0 for case $ES_1$. Initial
  conditions in the white region lead to unbounded
  orbits, those in the light blue
  region lead to the strange
  attractor whose cross
  section is in black, and those
  in the green region lead to
  the plane equilibrium
  }
\end{figure}

Figure \ref{fig:sur_fig3} shows a cross section in the xz plane at
$y = 0$ of the basin of attraction for the two attractors
for the typical case $ES_1$. The basin of attraction of the
chaotic attractor intersects the plane equilibrium almost
nowhere except in a tiny area near $x = 0, z = 1$. Thus,
the strange attractor is hidden in the sense that there are
uncountable points on the equilibrium plane of which a
tiny fraction intersect the basin of the chaotic attractor.
In other words, for computational purposes, the attractor
is hidden from the equilibria to some extent, and
knowledge about the equilibrium plane will not guarantee
that its location can be found.

As in the cases with a line of equilibrium points,
the strange attractor can reside very close to the surface
of equilibria. However, there is a fundamental difference
between the two cases. In a three-dimensional
system, the attractor can surround the equilibrium line,
and often does, whereas a surface equilibrium divides
the space into two regions that the attractor cannot span
because the normal component of the flow is zero at the
surface. Our experience shows that the surface equilibrium
can be and often is very close to the strange attractor,
and that the strange attractor is little altered if the
surface is removed by eliminating the factor $f(x,y,z)$
in the equations. Furthermore, it is possible to have separate
strange attractors in the two regions (or more if
there are multiple surfaces), although we do not show
any such examples here.

There can be some practical uses for such systems.
A surface of equilibrium (such as a sphere) can act like
a protector shield for the strange attractor, causing no
entrance and exit. No perturbation smaller than that
radius of the sphere can cause unbounded solutions.
With such closed surfaces, the attractor will be hidden
in an egg, which makes finding it more difficult. This
is important in secure communication uses of chaotic
systems. In the next section, we investigate the feasibility
of constructing a real electrical circuit based on
the proposed systems.

\section{Conclusion}

In conclusion, it is apparent that simple chaotic systems
with surfaces of equilibria that seemed to be rare may
in fact be rather common. These systems belong to the
newly introduced class of chaotic systems with hidden
attractors and have not been previously described. Furthermore,
the study of chaotic flows with surfaces of
equilibria provides a good reference for building systems
with attractors that are protected from external
influences, which can increase the safety of engineering.

\bibliographystyle{spphys}
\bibliography{chaos_sur_equi}
