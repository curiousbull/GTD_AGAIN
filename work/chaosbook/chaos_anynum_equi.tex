\chapter{A Chaotic System with Any Number of Equilibria}
\label{chap:anynum_equi}

\begin{pauthor}
  Xiong Wang \\
  Institute for Advanced Study,Shenzhen University,Nanshan District Shenzhen, Guangdong, China 518060\\
  \email{wangxiong8686@szu.edu.cn}\\
  J.C. Sprott\\
  Department of Physics, University of Wisconsin, Madison, WI 53706, USA\\
  Guanrong Chen\\
  Department of Electronic,Engineering, City University of Hong Kong, Hong Kong SAR, China.\\
  \email{eegchen@cityu.edu.hk}\\
  Sajad Jafari\\
Biomedical Engineering Department, Amirkabir University of Technology, Tehran
15875–4413, Iran\\
  \email{sajadjafari83@gmail.com}
\end{pauthor}

\abstract{In the chaotic Lorenz system, Chen system and Rossler system, their equilibria are unstable
and the number of the equilibria are no more than three. This chapter shows how to construct some simple
chaotic systems that can have any preassigned number of equilibria. A chaotic system with no equilibrium
has been presented and discussed before. A methodology is presented in this chapter by adding
symmetry to a new chaotic system with only one stable equilibrium, to show that chaotic systems with any preassigned number of equilibria
can be generated. By adjusting the only parameter in these systems, one can further control the stability of
their equilibria. This result reveals an intrinsic relationship of the global dynamical behaviors with
the number and stability of the equilibria of a chaotic system.}


\section{Introduction}
\label{sec:any_intro}

In chaos theory, it is important to study the stability of the equilibria of an autonomous dynamical system.
For a dynamical system described by a set of autonomous ordinary differential equations (ODEs),
\(\dot{\textbf{x}} = f(\textbf{x}), \textbf{x}\in R^n\), if \(f(\textbf{x}_e)=0\) has real solution then \(x_e\) is called the equilibrium of this
dynamical system. An equilibrium is said to be hyperbolic if all eigenvalues of the system
Jacobian matrix have nonzero real parts. A hyperbolic equilibrium for three-dimensional (3D) autonomous system can be a node, saddle,
node-focus, or saddle-focus. For 3D autonomous hyperbolic type of dynamical systems,
a commonly accepted criterion for proving
the existence of chaos is due to Shil'nikov's \cite{Ovsyannikov1987On,L1970Sil,Shil2011Normal,Shilnikov1998Methods,BAOYING2011AN}.

It has also been noticed that although most chaotic systems are of hyperbolic type, there are still many
others that are not so. For nonhyperbolic type of chaotic systems, they usually do not have saddle-focus
equilibria, such as those found by Sprott \cite{Sprott1993Automatic,Sprott1994Some,Sprott1997Simplest,Sprott2000Algebraically}.

The well-known Lorenz system \cite{Lorenz1962Deterministic} and also Chen system \cite{AttractorYET,TETSUSHI2012BIFURCATION} and some other
Lorenz-like systems \cite{Zhou2005Š,Li2007Hopf,Li2011Dynamical,Mu2011On} all have two unstable saddle-foci and
one unstable node. They can generate a two-wing butterfly-shaped chaotic attractor. Near the center of the two wings, there lies one an unstable saddle-focus.
In recent years, some chaotic systems are found which can generate three-wing, four-wing, and even multi-wing attractors. Observe that, typically, for those
symmetrical four-wing attractors, near the center of their wings there also lies one unstable saddle-focus. This common feature may imply that the number of
equilibria basically determines the shape of a multiwing attractor. Therefore, it is interesting to ask: Is it possible to generate a chaotic system with an
arbitrarily preassigned number of equilibria? Is the number of equilibria always determinate the shape of an attractor?

Furthermore, regarding the stability of the equilibria, recall that recently Yang and Chen found a
chaotic system with one saddle and two stable node-foci \cite{QIGUI2008A}, and an unusual 3D autonomous quadratic
Lorenz-like chaotic system with only two stable node-foci \cite{QIGUI2010AN}. Moreover, Wang and Chen found
an interesting chaotic system with only one stable node-focus \cite{Wang2011A}. Thus, another interesting question is
whether it is possible for a chaotic system to have two, or three, or even an arbitrarily large number of stable/unstable equilibria?

\section*{A Modified Sprott E System with One Stable Equilibrium}
\label{sec:org4cbc203}
As mentioned before, a chaotic system with only one equilibrium, a saddle-focus can be generated by adjusting
Sprott E system,

\begin{equation}
\label{eq:any_equi_01}
\left\{
    \begin{array}{l}
      \dot{x} = yz + a \\
      \dot{y} = x^2-y \\
      \dot{z} = 1-4x
    \end{array}
  \right.
\end{equation}

When \(a=0\), it is the Sprott E system \cite{Sprott1997Simplest}; when \(a\neq0\), however, the stability of the single
equilibrium is fundamentally different.

Specifically, when \(a>0\), system (\ref{eq:any_equi_01}) possesses only one stable equilibrium:

\begin{equation}
  P \left( x_{E},y_{E},z_{E} \right)
  =\left(\frac{1}{4},\frac{1}{16},-16\,a\right)\,.
\end{equation}

In the following, this system \ref{eq:any_equi_01} is further modified
by imposing some kind of symmetry onto it, to have different numbers of equilibria while keeping this system chaotic.

\section{Chaotic system with two equilibria}
\label{sec:any_two}
Rewrite system (\ref{eq:any_equi_01}) in terms of \(u,v,w\) as follows:

\begin{equation}
    \begin{array}{l}
      \dot{u}=vw+a\\
      \dot{v}=u^{2}-v \\
      \dot{w}=1-4u\,.
    \end{array}
  \right.
\end{equation}

From Fig. \ref{fig:one_equi_01}, one can see that the $y$-axis does not intersect with the attractor in system (\ref{eq:any_equi_01}).
So, one may try to add a $y$-axis rotation symmetry to this system, as detailed below.

Consider the following simple coordinate transformation:

\begin{equation}
\label{eq:org071d103}
    \begin{array}{l}
      u = x^2-z^2\\
      v = y\\
      w = 2xz\,.
    \end{array}
  \right.
\end{equation}

This transformation can add a $y$-axis rotation symmetry, \(\mathbb{R}_{y}(\pi)\), to the original system, because for each \((u,v,w)\)
there are two points \((\pm x,\pm y,\pm z)\) corresponding to \((u,v,w)\).

After the above transformation, the system becomes

\begin{equation}
\label{eq:any_equi_02}
    \begin{array}{l}
      \dot{x}=\frac{1}{2}{\frac{z+2y{x}^{2}z+xa-4{x}^{2}z+4{z}^{3}}{{x}^{2}+{z}^{2}}}\\
      \dot{y}=\left({x}^{2}-{z}^{2} \right) ^{2}-y\\
      \dot{z}=-\frac{1}{2}{\frac{2yx{z}^{2}+za-4x{z}^{2}-x+4{x}^{3}}{{x}^{2}+{z}^{2}}}\,.
    \end{array}
  \right.
\end{equation}

The new system (\ref{eq:any_equi_02}) possesses two symmetrical equilibria, which are independent of the parameter \(a\):
\(P1(\frac{1}{2},\frac{1}{16},0)\) and \(P1(-\frac{1}{2},\frac{1}{16},0)\).

System (\ref{eq:any_equi_02}) is not globally but only locally topologically equivalent to the original system (\ref{eq:any_equi_01}), 
however. Yet one can control the stability of these equilibria by
adjusting the parameter \(a\), so that the stability remains the same as the original system (\ref{eq:any_equi_01}) which, when \(a<0\) are unstable and when \(a>0\) are stable.

By linearizing system (\ref{eq:any_equi_01}) at
\(P1(\frac{1}{2},\frac{1}{16},0)\), one obtains the Jacobian

\begin{eqnarray}
  J\,\Big|_O =\left[
  \begin{array}{ccc}
    -2a & 0 & \frac{1}{16} \\
    \frac{1}{2} & -1 & 0 \\
    -4 & 0 & -2a \\
  \end{array}
  \right]\,,
\end{eqnarray}

whose characteristic equation is

\begin{equation} 
   {\rm det}(\lambda I-J|_O)
   =\lambda^3+(1+4a)\lambda^2+\Big(4a^2+4a+\frac{1}{4}\Big)\lambda
   + 4a^2 + \frac{1}{4}=0\,,
\end{equation}

which yields

\begin{eqnarray}
  \lambda_1&=&-1<0,\nonumber\\
  \lambda_2&=&-2a+0.5i,\nonumber\\
  \lambda_3&=&-2a-0.5i\,.\nonumber
\end{eqnarray}

System (\ref{eq:any_equi_02}) can generate a symmetrical two-petal chaotic attractor, as shown in Fig. \ref{fig:any_equi_01} and Fig. \ref{fig:any_equi_02}, respectively.

Numerical calculation of the largest Lyapunov exponent of the the system indicates the existence of chaos for some particular values of parameter \(a\), as 
shown in Fig. \ref{fig:any_equi_03}.

\begin{figure}[htbp]
\centering
\includegraphics[width=10cm]{chaos/2p0005.eps}
\caption{\label{fig:any_equi_01}
 The new two-petal chaotic attractor with stable equilibria when \(a=0.003\)}
\end{figure}

\begin{figure}[htbp]
\centering
\includegraphics[width=10cm]{chaos/2p-001.eps}
\caption{\label{fig:any_equi_02}
 The new two-petal chaotic attractor with unstable equilibria when \(a=-0.01\).}
\end{figure}

\begin{figure}[htbp]
\centering
\includegraphics[width=10cm]{chaos/2lle.eps}
\caption{\label{fig:any_equi_03}
 The largest Lyapunov exponent of system (\ref{eq:any_equi_02}) with respect to parameter \(a\).}
\end{figure}

\subsection{Chaotic System with Three Equilibria}

Similarly, consider the following transformation:
\begin{equation}\label{tran3}
\begin{array}{l}
u = x^3-3xz^2\\
v = y\\
w = 3x^2z-z^3\,.
\end{array}
\right.
\end{equation} %
It can add a $y$-axis rotation symmetry, $\mathbb{R}_{y}(\frac{2}{3}\pi)$, to the original system, because for each $(u,v,w)$ there are three symmetrical points corresponding to it.

After the above transformation, the system becomes
\begin{equation}\hspace{-20pt}\label{chaos3}
\left\{\hspace{-5pt}
\begin{array}{l}
\dot{x}=\frac{1}{3}{\frac
{3\,{x}^{4}zy-4\,{x}^{2}{z}^{3}y+{x}^{2}a-8\,{x}^{4}z+2\,z
x+24\,{x}^{2}{z}^{3}+{z}^{5}y-{z}^{2}a}{2\,{x}^{2}{z}^{2}+{x}^{4}+{z}^
{4}}} \\
\dot{y}=(x^3-3xz^2)^2-y \\
\dot{z}=-\frac{1}{3}{\frac
{6\,{z}^{2}{x}^{3}y-2\,{z}^{4}xy+2\,zxa+4\,{x}^{5}-{x}^{2}-16\,
{z}^{2}{x}^{3}+{z}^{2}+12\,{z}^{4}x}{2\,{x}^{2}{z}^{2}+{x}^{4}+{z}^{4}}}\,.
\end{array}
\right.
\end{equation}\label{eq3}

This system possesses three symmetrical equilibria, which are
dependent on the parameter $a$. The analytical expression are too
long to write out here, so only some numerical results are shown in Table \ref{3e}.

\begin{table}[htbp]\caption{Equilibria and Jacobian eigenvalues of system (\ref{chaos3})}
  \centering
\begin{tabular}{|c|c|p{150pt}|c|}
\hline & $a$ & Equilibria &
Jacobian eigenvalues \\
\hline Unstable case & 
\, $-0.01$\, & P1=(0.6550,0.0625,0.1258) \par
P2=(-0.2186,0.0625,-0.6300) \par P3=(-0.4365,0.0625,0.5044)&
$-1.0617,0.0308\pm 0.4843i$\, \\
\hline Stable case & 
\, 0.01 & P1=(0.6550,0.0625,-0.1258) \par
P2=(-0.2186,0.0625,0.6300) \par P3=(-0.4365,0.0625,-0.5044) & $-0.9334,-0.0333\pm 0.5165i$ \, \\
\hline
\end{tabular}
\label{3e}
\end{table}

The system has a symmetrical three-petal chaotic attractor, as shown in Fig. \ref{3s} and Fig. \ref{3us}, respectively. 

Numerical calculation of the largest Lyapunov exponent indicates the existence of chaos for some particular values of parameter $a$, as shown in Fig. \ref{3lle}.

\begin{figure}
\centering
\includegraphics[width=10cm]{chaos/3p001}
\caption{Chaotic attractor of system (\ref{chaos3}) with three stable symmetrical equilibria when $a=0.01$.}\label{3s}
\end{figure}

\begin{figure}
\centering
\includegraphics[width=10cm]{chaos/3p-001}
\caption{Chaotic attractor of system (\ref{chaos3}) with three unstable symmetrical equilibria when $a=-0.01$.}\label{3us}
\end{figure}

\begin{figure}
%\centering
\includegraphics[width=10cm]{chaos/3lle}
\caption{The largest Lyapunov exponent of system (\ref{chaos3}) with respect to the parameter $a$. }\label{3lle}
\end{figure}

\section{Chaotic System with Any Number of Equilibria}

Theoretically, one can use the transform $(x+iz)^{n}=(u+iv)$ to
obtain a new chaotic system with $n$ equilibria. The following is such an example with five equilibria.

Consider the following transformation:
\begin{equation}\label{tran5}
\left\{
\begin{array}{l}
u = x^5-10x^3z^2+5xz^4\\
v = y\\
w = 5x^4z-10x^2z^3+z^5\,.
\end{array}
\right.
\end{equation} %
It can add a $y$-axis rotation symmetry, $\mathbb{R}_{y}(\frac{2}{5}\pi)$, to the original system. The new equations are too long to write out, so are omitted here. The symmetrical attractor generated with parameter $a=0$ is shown in in Fig. \ref{5}.

\begin{figure}
\centering
\includegraphics[width=8cm]{chaos/5p}
\caption{Chaotic attractor of the new system with five symmetrical equilibria when $a=0$.}\label{5}
\end{figure}

\section{Discussions}
The Hartman-Grobman Theorem is an important theorem in ODE systems theory. It is
about the \emph{local\,} behavior
of an autonomous dynamical system in the neighborhood of a hyperbolic
equilibrium,
stating that the behavior of the dynamical system near the hyperbolic
equilibrium is
qualitatively the same as (i.e., topologically equivalent to the behavior of its linearization near this equilibrium point.

Notice, however, that the new systems discussed in this chapter have chaos,
which is a \emph{global\,} behavior, although such a system
has only one hyperbolic equilibrium point. In other words, all system flows
locally converge
to the stable equilibrium, but they are chaotic globally. This interesting
phenomenon is
worth further studying, both theoretically and experimentally, to further reveal
the intrinsic relationship between the local stability of an equilibrium and the
global complex dynamical behaviors of a chaotic system.

