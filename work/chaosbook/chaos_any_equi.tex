\chapter{A Chaotic System with Any Number and Any Type of Equilibria}
\label{chap:any_equi}
\begin{pauthor}
  Xiong Wang \\
  Institute for Advanced Study,Shenzhen University,Nanshan District Shenzhen, Guangdong, China 518060\\
  \email{wangxiong8686@szu.edu.cn}\\
  J.C. Sprott\\
  Department of Physics, University of Wisconsin, Madison, WI 53706, USA\\
  Guanrong Chen\\
  Department of Electronic,Engineering, City University of Hong Kong, Hong Kong SAR, China.\\
  \email{eegchen@cityu.edu.hk}\\
  Sajad Jafari\\
Biomedical Engineering Department, Amirkabir University of Technology, Tehran
15875–4413, Iran\\
  \email{sajadjafari83@gmail.com}
\end{pauthor}

\section{A Chaotic System with Any Number of Equilibria}

\abstract{In the chaotic Lorenz system, Chen system and Rossler system, their equilibria are unstable
and the number of the equilibria are no more than three. This chapter shows how to construct some simple
chaotic systems that can have any preassigned number of equilibria. A chaotic system with no equilibrium
has been presented and discussed before. A methodology is presented in this chapter by adding symmetry to a
new chaotic system with only one stable equilibrium, to show that chaotic systems with any preassigned number of equilibria
can be generated. By adjusting the only parameter in these systems, one can further control the stability of
their equilibria. This result reveals an intrinsic relationship of the global dynamical behaviors with
the number and stability of the equilibria of a chaotic system.}


\subsection{Introduction}
\label{sec:any_intro}

In chaos theory, it is important to study the stability of the equilibria of an autonomous dynamical system.
For a dynamical system described by a set of autonomous ordinary differential equations (ODEs),
\(\dot{x} = f(x), x\in R^n\), if \(f(x_e)=0\) has real solution then \(x_e\) is called the equilibrium of this
dynamical system. An equilibrium is said to be hyperbolic if all eigenvalues of the system Jacobian matrix
have nonzero real parts. A hyperbolic equilibrium for three-dimensional (3D) autonomous system can be a node, saddle,
node-focus, or saddle-focus. For 3D autonomous hyperbolic type of dynamical systems, a commonly accepted criterion for proving
the existence of chaos is due to Silnikov \cite{Ovsyannikov1987On,L1970Sil,Shil2011Normal,Shilnikov1998Methods,BAOYING2011AN}.

It has also been noticed that although most chaotic systems are of hyperbolic type, there are still many
others that are not so. For nonhyperbolic type of chaotic systems, they usually do not have saddle-focus
equilibria, such as those found by Sprott \cite{Sprott1993Automatic,Sprott1994Some,Sprott1997Simplest,Sprott2000Algebraically}.

The well-known Lorenz system \cite{Lorenz1962Deterministic} and also Chen system \cite{AttractorYET,TETSUSHI2012BIFURCATION} and some other
Lorenz-like systems \cite{Zhou2005Š,Li2007Hopf,Li2011Dynamical,Mu2011On} all have two unstable saddle-foci and
one unstable node. They can generate a two-wing butterfly-shaped chaotic attractor. Near the center of the two wings, there lies one an unstable saddle-focus.
In recent years, some chaotic systems are found which can generate three-wing, four-wing, and even multi-wing attractors. Observe that, typically, for those
symmetrical four-wing attractors, near the center of their wings there also lies one unstable saddle-focus. This common feature may imply that the number of
equilibria basically determines the shape of a multiwing attractor. Therefore, it is interesting to ask: Is it possible to generate a chaotic system with an
arbitrarily preassigned number of equilibria? Is the number of equilibria always determinate the shape of an attractor?

Furthermore, regarding the stability of the equilibria, recall that recently Yang and Chen found a
chaotic system with one saddle and two stable node-foci \cite{QIGUI2008A}, and an unusual 3D autonomous quadratic
Lorenz-like chaotic system with only two stable node-foci \cite{QIGUI2010AN}. Moreover, Wang and Chen found
an interesting chaotic system with only one stable node-focus \cite{Wang2011A}. Thus, another interesting question is
whether it is possible for a chaotic system to have two, or three, or even an arbitrarily large number of stable/unstable equilibria?

\subsubsection*{A Modified Sprott E System with One Stable Equilibrium}
\label{sec:org4cbc203}
As mentioned before, a chaotic system with only one equilibrium, a saddle-focus can be generated by adjusting
Sprott E system,

\begin{equation}
\label{eq:any_equi_01}
    \begin{array}
      \dot{x} = yz + a \\
      \dot{y} = x^2-y \\
      \dot{z} = 1-4x
    \end{array}
  \right.
\end{equation}

When \(a=0\), it is the Sprott E system \cite{Sprott1997Simplest}; when \(a\neq0\), however, the stability of the single
equilibrium is fundamentally different.

Specifically, when \(a>0\), system (\ref{eq:any_equi_01}) possesses only one stable equilibrium:

\begin{equation}
  P \left( x_{E},y_{E},z_{E} \right)
  =\left(\frac{1}{4},\frac{1}{16},-16\,a\right)\,.
\end{equation}

In the following, this system \ref{eq:any_equi_01} is further modified
by imposing some kind of symmetry onto it, to have different numbers of equilibria while keeping this system chaotic.

\subsection{Chaotic system with two equilibria}
\label{sec:any_two}
Rewrite system (\ref{eq:any_equi_01}) in terms of \(u,v,w\) as follows:

\begin{equation}
    \begin{array}{l}
      \dot{u}=vw+a\\
      \dot{v}=u^{2}-v \\
      \dot{w}=1-4u\,.
    \end{array}
  \right.
\end{equation}

From Fig. \ref{fig:one_equi_01}, one can see that the $y$-axis does not intersect with the attractor in system (\ref{eq:any_equi_01}).
So, one may try to add a $y$-axis rotation symmetry to this system, as detailed below.

Consider the following simple coordinate transformation:

\begin{equation}
\label{eq:org071d103}
    \begin{array}{l}
      u = x^2-z^2\\
      v = y\\
      w = 2xz\,.
    \end{array}
  \right.
\end{equation}

This transformation can add a $y$-axis rotation symmetry, \(\mathbb{R}_{y}(\pi)\), to the original system, because for each \((u,v,w)\)
there are two points \((\pm x,\pm y,\pm z)\) corresponding to \((u,v,w)\).

After the above transformation, the system becomes

\begin{equation}
\label{eq:any_equi_02}
    \begin{array}{l}
      \dot{x}=\frac{1}{2}{\frac{z+2y{x}^{2}z+xa-4{x}^{2}z+4{z}^{3}}{{x}^{2}+{z}^{2}}}\\
      \dot{y}=\left({x}^{2}-{z}^{2} \right) ^{2}-y\\
      \dot{z}=-\frac{1}{2}{\frac{2yx{z}^{2}+za-4x{z}^{2}-x+4{x}^{3}}{{x}^{2}+{z}^{2}}}\,.
    \end{array}
  \right.
\end{equation}

The new system (\ref{eq:any_equi_02}) possesses two symmetrical equilibria, which are independent of the parameter \(a\):
\(P1(\frac{1}{2},\frac{1}{16},0)\) and \(P1(-\frac{1}{2},\frac{1}{16},0)\).

System (\ref{eq:any_equi_02}) is not globally but only locally topologically equivalent to the original system (\ref{eq:any_equi_01}), 
however. Yet one can control the stability of these equilibria by
adjusting the parameter \(a\), so that the stability remains the same as the original system (\ref{eq:any_equi_01}) which, when \(a<0\) are unstable and when \(a>0\) are stable.

By linearizing system (\ref{eq:any_equi_01}) at
\(P1(\frac{1}{2},\frac{1}{16},0)\), one obtains the Jacobian

\begin{eqnarray}
  J\,\Big|_O =\left[
  \begin{array}{ccc}
    -2a & 0 & \frac{1}{16} \\
    \frac{1}{2} & -1 & 0 \\
    -4 & 0 & -2a \\
  \end{array}
  \right]\,,
\end{eqnarray}

whose characteristic equation is

\begin{equation} 
   {\rm det}(\lambda I-J|_O)
   =\lambda^3+(1+4a)\lambda^2+\Big(4a^2+4a+\frac{1}{4}\Big)\lambda
   + 4a^2 + \frac{1}{4}=0\,,
\end{equation}

which yields

\begin{eqnarray}
  \lambda_1&=&-1<0,\nonumber\\
  \lambda_2&=&-2a+0.5i,\nonumber\\
  \lambda_3&=&-2a-0.5i\,.\nonumber
\end{eqnarray}

System (\ref{eq:any_equi_02}) can generate a symmetrical two-petal chaotic attractor, as shown in Fig. \ref{fig:any_equi_01} and Fig. \ref{fig:any_equi_02}, respectively.

Numerical calculation of the largest Lyapunov exponent of the the system indicates the existence of chaos for some particular values of parameter \(a\), as 
shown in Fig. \ref{fig:any_equi_03}.

\begin{figure}[htbp]
\centering
\includegraphics[width=10cm]{chaos/2p0005.eps}
\caption{\label{fig:any_equi_01}
 The new two-petal chaotic attractor with stable equilibria when \(a=0.003\)}
\end{figure}

\begin{figure}[htbp]
\centering
\includegraphics[width=10cm]{chaos/2p-001.eps}
\caption{\label{fig:any_equi_02}
 The new two-petal chaotic attractor with unstable equilibria when \(a=-0.01\).}
\end{figure}

\begin{figure}[htbp]
\centering
\includegraphics[width=10cm]{chaos/2lle.eps}
\caption{\label{fig:any_equi_03}
 The largest Lyapunov exponent of system (\ref{eq:any_equi_02}) with respect to parameter \(a\).}
\end{figure}

\subsection{Chaotic System with Three Equilibria}

Similarly, consider the following transformation:
\begin{equation}\label{tran3}
\begin{array}{l}
u = x^3-3xz^2\\
v = y\\
w = 3x^2z-z^3\,.
\end{array}
\right.
\end{equation} %
It can add a $y$-axis rotation symmetry, $\mathbb{R}_{y}(\frac{2}{3}\pi)$, to the original system, because for each $(u,v,w)$ there are three symmetrical points corresponding to it.

After the above transformation, the system becomes
\begin{equation}\hspace{-20pt}\label{chaos3}
\left\{\hspace{-5pt}
\begin{array}{l}
\dot{x}=\frac{1}{3}{\frac
{3\,{x}^{4}zy-4\,{x}^{2}{z}^{3}y+{x}^{2}a-8\,{x}^{4}z+2\,z
x+24\,{x}^{2}{z}^{3}+{z}^{5}y-{z}^{2}a}{2\,{x}^{2}{z}^{2}+{x}^{4}+{z}^
{4}}} \\
\dot{y}=(x^3-3xz^2)^2-y \\
\dot{z}=-\frac{1}{3}{\frac
{6\,{z}^{2}{x}^{3}y-2\,{z}^{4}xy+2\,zxa+4\,{x}^{5}-{x}^{2}-16\,
{z}^{2}{x}^{3}+{z}^{2}+12\,{z}^{4}x}{2\,{x}^{2}{z}^{2}+{x}^{4}+{z}^{4}}}\,.
\end{array}
\right.
\end{equation}\label{eq3}

This system possesses three symmetrical equilibria, which are
dependent on the parameter $a$. The analytical expression are too
long to write out here, so only some numerical results are shown in Table \ref{3e}.

\begin{table}[htbp]\caption{Equilibria and Jacobian eigenvalues of system (\ref{chaos3})}
  \centering
\begin{tabular}{|c|c|p{150pt}|c|}
\hline & $a$ & Equilibria &
Jacobian eigenvalues \\
\hline Unstable case & 
\, $-0.01$\, & P1=(0.6550,0.0625,0.1258) \par
P2=(-0.2186,0.0625,-0.6300) \par P3=(-0.4365,0.0625,0.5044)&
$-1.0617,0.0308\pm 0.4843i$\, \\
\hline Stable case & 
\, 0.01 & P1=(0.6550,0.0625,-0.1258) \par
P2=(-0.2186,0.0625,0.6300) \par P3=(-0.4365,0.0625,-0.5044) & $-0.9334,-0.0333\pm 0.5165i$ \, \\
\hline
\end{tabular}
\label{3e}
\end{table}

The system has a symmetrical three-petal chaotic attractor, as shown in Fig. \ref{3s} and Fig. \ref{3us}, respectively. 

Numerical calculation of the largest Lyapunov exponent indicates the existence of chaos for some particular values of parameter $a$, as shown in Fig. \ref{3lle}.

\begin{figure}
\centering
\includegraphics[width=10cm]{chaos/3p001}
\caption{Chaotic attractor of system (\ref{chaos3}) with three stable symmetrical equilibria when $a=0.01$.}\label{3s}
\end{figure}

\begin{figure}
\centering
\includegraphics[width=10cm]{chaos/3p-001}
\caption{Chaotic attractor of system (\ref{chaos3}) with three unstable symmetrical equilibria when $a=-0.01$.}\label{3us}
\end{figure}

\begin{figure}
%\centering
\includegraphics[width=10cm]{chaos/3lle}
\caption{The largest Lyapunov exponent of system (\ref{chaos3}) with respect to the parameter $a$. }\label{3lle}
\end{figure}

\subsection{Chaotic System with Any Number of Equilibria}

Theoretically, one can use the transform $(x+iz)^{n}=(u+iv)$ to
obtain a new chaotic system with $n$ equilibria. The following is such an example with five equilibria.

Consider the following transformation:
\begin{equation}\label{tran5}
\left\{
\begin{array}{l}
u = x^5-10x^3z^2+5xz^4\\
v = y\\
w = 5x^4z-10x^2z^3+z^5\,.
\end{array}
\right.
\end{equation} %
It can add a $y$-axis rotation symmetry, $\mathbb{R}_{y}(\frac{2}{5}\pi)$, to the original system. The new equations are too long to write out, so are omitted here. The symmetrical attractor generated with parameter $a=0$ is shown in in Fig. \ref{5}.

\begin{figure}
\centering
\includegraphics[width=8cm]{chaos/5p}
\caption{Chaotic attractor of the new system with five symmetrical equilibria when $a=0$.}\label{5}
\end{figure}

\subsection{Discussions}
The Hartman-Grobman Theorem is an important theorem in ODE systems theory. It is
about the \emph{local\,} behavior
of an autonomous dynamical system in the neighborhood of a hyperbolic
equilibrium,
stating that the behavior of the dynamical system near the hyperbolic
equilibrium is
qualitatively the same as (i.e., topologically equivalent to the behavior of its linearization near this equilibrium point.

Notice, however, that the new systems discussed in this chapter have chaos,
which is a \emph{global\,} behavior, although such a system
has only one hyperbolic equilibrium point. In other words, all system flows
locally converge
to the stable equilibrium, but they are chaotic globally. This interesting
phenomenon is
worth further studying, both theoretically and experimentally, to further reveal
the intrinsic relationship between the local stability of an equilibrium and the
global complex dynamical behaviors of a chaotic system.


\section{A Chaotic System with Any Type of Equilibria}

\abstract{Of the eight types of hyperbolic equilibrium points in threedimensional
flows, one is overwhelmingly dominant in dissipative
chaotic systems. This chapter also shows examples of chaotic systems for
each of the eight types as well as one without any equilibrium and
two that are nonhyperbolic. The systems are a generalized form of the
Nos´e–Hoover oscillator with a single equilibrium point. Six of the eleven
cases have hidden attractors, and six of them exhibit multistability for
the chosen parameters.}

\subsection{Introduction}
Nearly all the common examples of strange attractors in three-dimensional autonomous
systems of ordinary differential equations occur in systems for which there
are one or more spiral saddle points, where the Shilnikov condition is satisfied \cite{01Shilnikov1998Methods}. Such
attractors have been called “self-excited” since they can be found by choosing initial
conditions on the unstable manifold in the vicinity of one of these equilibria \cite{02Leonov2011Localization,03Leonov2012IWCFTA2012}.

It is natural to ask whether strange attractors can exist in systems with only one
equilibrium of the other eighteen types that can occur in such systems. A system that
admits a wide variety of chaotic solutions is a generalized version of the Nos´e–Hoover
oscillator \cite{04shuichi1991constant,05Hoover1995Remark} given by
\begin{equation}
\label{eq:anytype_equi_01}
    \begin{array}
      \dot{x} = y \\
      \dot{y} = -x+yz \\
      \dot{z} = f(x,y,z) 
    \end{array}
  \right.
\end{equation}
where
\begin{equation}
\label{eq:anytype_equi_02}
f(x, y, z) = a_1x + a_2y + a_3z + a_4x^2 + a_5y^2 + a_6xy + a_7xz + a_8yz + a_9
\end{equation}
is the most general function $f(x, y, z)$ with quadratic nonlinearities that is guaranteed
to provide a single equilibrium point at $(x, y, z) = (0, 0, −a_9/a_3)$ provided $a_3 = 0$ or
$a_9 = 0$. An equilibrium point, sometimes called a fixed point or singular point, is a
value of $(x, y, z)$ where $\dot{x} = \dot{y} = \dot{z} = 0$. Note that it is also possible to include a $z^2$
term in Eq. (\ref{eq:anytype_equi_02}) and adjust the constant $a_9$ so that the two roots of $f(x, y, z)=0$
coincide, but that turns out to be unnecessary for the present purpose.

The simplest chaotic system of this form has $f = 1 − y^2$ and has been much
studied, but it is a conservative system without any equilibrium points. However,
there is a weakly dissipative form of System (\ref{eq:anytype_equi_01}) with $f
= 1+\epsilon{}tanh x − y^2$ that has
a strange attractor but with no equilibrium points since $a_3 = 0$ and $a_9 =0$\cite{06Sprott2014Heat}. Note
that all systems in the form of Eq. (\ref{eq:anytype_equi_01}) have a nonlinear damping whose magnitude
and sign depend on the time average of z along the orbit because of the yz term.

System (\ref{eq:anytype_equi_01}) with various forms of $f(x, y, z)$ has previously been shown to have
strange attractors for two of the equilibrium types
\cite{07MALIHE2013SIMPLE,08Sprott2015A}. The goal of this chapter is to
give simple examples of the remaining types, which have not been previously reported.
This involves putting appropriate constraints on the eigenvalues $\lambda$ given by
\begin{equation}
\label{eq:anytype_equi_03}
\lambda^3 +\left( \frac{a_9}{a_3}−a_3 \right)\lambda^2 + (1 − a_9)λ − a3 = 0
\end{equation}
and searching the nine-dimensional parameter space and three-dimensional space
of initial conditions for solutions that are bounded and dissipative with a positive
Lyapunov exponent and then simplifying the resulting systems to the extent possible
by removing unnecessary terms and setting parameters to unity.

\subsection{System with no equilibrium}

Shown first is a simple example of a dissipative chaotic system in the form of Eq. (\ref{eq:anytype_equi_01})
with no equilibrium given by
\begin{equation}
\label{eq:anytype_equi_04}
    \begin{array}
      \dot{x} = y \\
      \dot{y} = -x+yz \\
      \dot{z} = x^2-4y^2+1
    \end{array}
  \right.
\end{equation}

For initial conditions of (0, 2, 0), it has a strange attractor with Lyapunov exponents of
(0.0131, 0, −0.0155) and a relatively large Kaplan–Yorke dimension of 2.8455 as shown
in Fig. \ref{fig:anytype_01}. This strange attractor is “hidden” in the sense that
it cannot be found by using an initial condition in the vicinity of an equilibrium point
since no such point exists \cite{02Leonov2011Localization,03Leonov2012IWCFTA2012}.

The system is time-reversal invariant under the transformation $(x, y, z, t)\rightarrow(x, −y, −z, −t)$.
The strange attractor is slightly asymmetric and is accompanied
by a strange repellor that is symmetric with it under a $180^{\circ}$ rotation about the xaxis
and intertwined with it. The attractor and repellor exchange roles when time is
reversed. The asymmetry in z is the source of the weak nonlinear damping, since its
value time-averaged along the chaotic orbit is $<z>\approx{}−0.0024$.

This system is unusual because the strange attractor is intertwined with a set of
nested conservative tori, which are symmetric under rotation about the x-axis and
have $<z> = 0$. Figure \ref{fig:anytype_01} shows one such torus for initial conditions of (0, 1.2, 0) for which
the Lyapunov exponents are (0, 0, 0). Figure \ref{fig:anytype_02} shows a cross section in the z = 0 plane
of the nested tori surrounded by what looks like a chaotic sea but is actually a weakly
dissipative strange attractor. Sixty-three initial conditions were spaced uniformly over
the range $−2.9625 \leq x \leq −0.6522$ with $y = z = 0$. The blue background shows the
initial conditions that give conservative orbits (tori), and the yellow background is
the basin of attraction for the strange attractor. It appears that there are additional
thin tori toward the outer edge of the strange attractor. The basin of attraction of
the strange attractor is the whole of state space except for the region of finite volume
occupied by tori. Only a few other such examples are known \cite{06Sprott2014Heat,09Politi1986Coexistence,10Sprott2014A}.

\begin{figure}[htbp]
\centering
\includegraphics[width=0.6\textwidth]{chaos/anytype_01.png}
\caption{\label{fig:anytype_01}
A strange attractor (in red) for System (\ref{eq:anytype_equi_03}) coexisting with a conservative invariant
torus (in green) projected onto the xy-plane.}
\end{figure}

\begin{figure}[htbp]
\centering
\includegraphics[width=0.6\textwidth]{chaos/anytype_02.png}
\caption{\label{fig:anytype_02}
Cross section in the z = 0 plane of the nested tori surrounded by a strange attractor
for System ET0. The blue background shows the initial conditions that give conservative
orbits (tori), and the yellow background is the basin of attraction for the strange attractor.}
\end{figure}

\subsection{Hyperbolic examples}
There are eight types of hyperbolic equilibrium points in three-dimensional flows as
shown in Fig. \ref{fig:anytype_03}. A hyperbolic equilibrium is one in which all eigenvalues have a
nonzero real part. Type 7 is overwhelmingly the most common in dissipative chaotic
systems, but examples also occur for the other seven cases as shown below.

\begin{figure}[htbp]
\centering
\includegraphics[width=0.6\textwidth]{chaos/anytype_03.png}
\caption{\label{fig:anytype_03}
 Types of hyperbolic equilibria in three-dimensional ODEs.}
\end{figure}

\subsubsection{Equilibrium Type 1 (index-0 node)}
This system has a single equilibrium point with three real eigenvalues, all negative.
Hence it is a stable node with index 0, where the index is the number of eigenvalues
with a positive real part, or, equivalently, the dimension of the unstable manifold.
A system of this type is

\begin{equation}
\label{eq:anytype_equi_05}
    \begin{array}
      \dot{x} = y \\
      \dot{y} = -x+yz \\
      \dot{z} = −z − 8xy + 0.3xz − 3.
    \end{array}
  \right.
\end{equation}

It has an equilibrium with eigenvalues given by $\lambda = (−0.381966, −1, −2.618034)$ and
a strange attractor with Lyapunov exponents of (0.0505, 0, −17.2283). The strange
attractor is hidden since all initial conditions in the vicinity of the equilibrium are
attracted to the equilibrium point. It is also multistable since the strange attractor
coexists with a point attractor. Initial conditions close to the attractor are (−2, 1, 0.7),
and the basin of attraction is very small.

\subsubsection{Equilibrium Type 2 (index-1 saddle point)}
This system has three real eigenvalues, with two negative and one positive. A system
of this type is
\begin{equation}
\label{eq:anytype_equi_06}
    \begin{array}
      \dot{x} = y \\
      \dot{y} = -x+yz \\
      \dot{z} = 0.5z − y^2 + 5.
    \end{array}
  \right.
\end{equation}
It has an equilibrium with eigenvalues given by $\lambda = (0.5, −0.101021, −9.898980)$ and
a symmetric pair of tightly intertwined strange attractors with Lyapunov exponents
of (0.0141, 0, −0.3030) that coexist with three limit cycles, a symmetric one with
Lyapunov exponents of (0, −0.1239, −0.2336) and a symmetric pair with Lyapunov
exponents of (0, −0.0264, −0.0264). This is possible because System (\ref{eq:anytype_equi_05}), like the
Lorenz system, has a rotational symmetry about the z-axis as evidenced by its invariance
under the transformation $(x, y, z) \rightarrow (−x, −y, z)$, and hence any solutions
either share that symmetry or there is a symmetric pair of them. All the attractors
are hidden since all initial conditions in the vicinity of the equilibrium point produce
unbounded orbits. Initial conditions that give the five attractors are $(\pm{}0.9, 0, −2)$,
(0.43, 2, 0.18), and $(\pm{}0.4, \pm{}3, 1)$, and the basins of attraction of the strange attractors
are relatively small and bounded (finite volume).

\subsubsection{Equilibrium Type 3 (index-2 saddle point)}
This system has three real eigenvalues, with two positive and one negative. A system
of this type is
\begin{equation}
\label{eq:anytype_equi_07}
  \left\{
    \begin{array}
      \dot{x} = y \\
      \dot{y} = -x+yz \\
      \dot{z} = −x − 0.1z − y^2 + 0.3.
    \end{array}
  \right.
\end{equation}
It has an equilibrium with eigenvalues given by $\lambda = (2.618034, 0.381966, −0.1)$  and
a self-excited strange attractor with Lyapunov exponents of (0.02191, 0, −0.3181).
Initial conditions close to the attractor are (0, 0.1, 0), and the basin of attraction is
relatively large.

\subsubsection{Equilibrium Type 4 (index-3 repellor)}
This system has three real eigenvalues, all positive, and hence it is an unstable node,
sometimes simply called a “repellor”. A system of this type has recently been reported
\cite{08Sprott2015A} and has the form
\begin{equation}
\label{eq:anytype_equi_08}
  \left\{
    \begin{array}
      \dot{x} = y \\
      \dot{y} = -x+yz \\
      \dot{z} = z + 8.894x^2 − y^2 − 4.
    \end{array}
  \right.
\end{equation}
It has an equilibrium with eigenvalues given by $\lambda = (3.732051, 1, 0.267949)$ and
a strange attractor with Lyapunov exponents of (0.1767, 0, −0.9158). The equations
have a rotational symmetry since they are invariant under the transformation
$(x, y, z) \rightarrow (−x, −y, z)$, and the system does have a symmetric pair of solutions for
some parameters, but not for the ones given above. The symmetric strange attractor
for this case is hidden since all initial conditions chosen in the vicinity of the
equilibrium lead to unbounded solutions. Initial conditions close to the attractor are
(0, 3.8, 0.7), and the basin of attraction is very small.

\subsubsection{Equilibrium Type 5 (index-0 spriral node)}
This system has one real negative eigenvalue, and a complex conjugate pair with a
negative real part. Twenty-three chaotic examples of this type have recently been
reported \cite{07MALIHE2013SIMPLE}. One typical case in the form of Eq. (\ref{eq:anytype_equi_01}) is
\begin{equation}
\label{eq:anytype_equi_09}
  \left\{
    \begin{array}
      \dot{x} = y \\
      \dot{y} = -x+yz \\
      \dot{z} = 2x − 2z + y^2 − 0.3.
    \end{array}
  \right.
\end{equation}
It has an equilibrium with eigenvalues given by $\lambda = (−2, −0.075 \pm 0.997184i)$ and a
strange attractor with Lyapunov exponents of (0.0203, 0, −2.4751). All chaotic systems
of this type are multistable since the strange attractor coexists with a stable
equilibrium point, and the strange attractor is hidden since it cannot be found by
using initial conditions in the vicinity of the equilibrium. Initial conditions close to
the strange attractor are (0.9, 0, 0.7), and the basin of attraction is very large.

\subsubsection{6 Equilibrium Type 6 (index-1 spiral saddle)}
This system has one real positive eigenvalue, and a complex conjugate pair with a
negative real part. A system of this type is
\begin{equation}
\label{eq:anytype_equi_10}
  \left\{
    \begin{array}
      \dot{x} = y \\
      \dot{y} = -x+yz \\
      \dot{z} = 0.28z − xy + 0.48.
    \end{array}
  \right.
\end{equation}
It has an equilibrium with eigenvalues given by $\lambda = (0.28, −0.857143 \pm 0.515079i)$ 
 and a symmetric pair of strange attractors with Lyapunov exponents of
(0.0677, 0, −1.5020). The equations have a rotational symmetry since they are invariant
under the transformation $(x, y, z) \rightarrow (−x, −y, z)$. The strange attractors are
hidden, and all initial conditions in the vicinity of the equilibrium point lead to unbounded
orbits. Initial conditions close to the attractors are $(0, \pm 4, 2)$, and the basins
of attraction are very small.

\subsubsection{Equilibrium Type 7 (index-2 spiral saddle)}
This system has one real negative eigenvalue, and a complex conjugate pair with
a positive real part. This is overwhelmingly the most common type with abundant
examples including the familiar Lorenz \cite{11Lorenz1963Deterministic} and R¨ossler \cite{12R1976An} systems, although they
have multiple equilibrium points. The simplest such system with a single equilibrium
point is the jerk system $\dddot{x} + 2.017\ddot{x} − \dot{x}^2 + x =0$\cite{13Sprott1997Simplest}. Other simple examples are
Sprott Cases I, J, L, N, and R \cite{14Sprott1994Some}. There are also a number of systems in the form
of Eq. (\ref{eq:anytype_equi_01}) that have not been studied including the following
\begin{equation}
\label{eq:anytype_equi_11}
  \left\{
    \begin{array}
      \dot{x} = y \\
      \dot{y} = -x+yz \\
      \dot{z} = −z + xy + 0.39.
    \end{array}
  \right.
\end{equation}
which is chosen because it is functionally the same as System ET6 with a rotational
symmetry but with different parameters. It has an equilibrium with eigenvalues given
by $\lambda = (−1, 0.195 \pm 0.980803i)$ and a symmetric pair of strange attractors with
Lyapunov exponents of (0.0820, 0, −0.6920). The strange attractors are self-excited,
although the equilibrium point lies on their basin boundary, and so which attractor is
found depends on exactly where in the vicinity of the equilibrium the initial conditions
are chosen. Initial conditions close to the attractors are $(\pm 1.4, \pm 1, 1)$, and the basins
of attraction are relatively large.

\subsubsection{Equilibrium Type 8 (index-3 spiral repellor)}
This system has one real positive eigenvalue, and a complex conjugate pair with a
positive real part. A system of this type is
\begin{equation}
\label{eq:anytype_equi_12}
  \left\{
    \begin{array}
      \dot{x} = y \\
      \dot{y} = -x+yz \\
      \dot{z} = 0.2z + 0.1y^2 − xy − 0.08.
    \end{array}
  \right.
\end{equation}
It has an equilibrium with eigenvalues given by $\lambda = (0.2, 0.2\pm 0.979796i)$ and a strange
attractor with Lyapunov exponents of (0.1083, 0, −3.2555). The equations have a
rotational symmetry since they are invariant under the transformation $(x, y, z)
\rightarrow(−x, −y, z)$,
and for the given parameters, the strange attractor is symmetric and
self-excited. Initial conditions close to the attractor are (1, −2, 0.4), and the basin of
attraction is very large.

\subsection{Nonhyperbolic examples}

A nonhyperbolic equilibrium point has one or more eigenvalues with a zero real part.
There are eleven such types in three-dimensional flows. Six of these have all eigenvalues
real and are of the form (0, −, −), (+, 0, −), (+, +, 0), (0, 0, −), (+, 0, 0), and
(0, 0, 0). Five have one real and a complex conjugate pair of eigenvalues, only two of
which have nonzero real eigenvalues. The stability of those systems that do not have
an eigenvalue with a positive real part cannot be determined from the eigenvalues
and requires a nonlinear analysis.
Consider first the nine types where at least one eigenvalue is real and zero. With
$\lambda = 0$ , Eq. (\ref{eq:anytype_equi_03}) shows that $a_3 = 0$ , and thus
according
to Eq. (\ref{eq:anytype_equi_03}), an equilibrium point
is present only if $a_9 = 0$, in which case there is an infinite line of equilibrium points
along the z-axis at (0, 0, z). Such cases have been previously studied \cite{15Jafari2013Simple} including
ones in the form of Eq. (\ref{eq:anytype_equi_01}), but they fall outside the scope of the present paper which
involves chaotic systems with a single equilibrium point. Thus nine of the eleven
possible nonhyperbolic isolated equilibrium points cannot occur in Eq. (\ref{eq:anytype_equi_01}), although
this does not imply that they cannot exist in other systems.
The remaining two cases have a complex conjugate pair of eigenvalues of the form
$\lambda = 0 \pm i\omega$. Substitution into Eq. (\ref{eq:anytype_equi_01}) gives $\omega_2 = 1−a_9$ and $a_9 = 1−a_3^2$
 or $a_9 = 0$.
The remaining real eigenvalue can be negative or positive. Chaotic examples of the
two types are given below.

\subsubsection{Equilibrium Type 9}
This system has a single equilibrium point with one real negative eigenvalue and a
complex conjugate pair with zero real parts. Chaotic systems of this type have been
reported such as Sprott Case E [14]. A system of this type in the form of Eq. (\ref{eq:anytype_equi_01}) is
\begin{equation}
\label{eq:anytype_equi_13}
  \left\{
    \begin{array}
      \dot{x} = y \\
      \dot{y} = -x+yz \\
      \dot{z} = −z − 4xy + xz.
    \end{array}
  \right.
\end{equation}
It has an equilibrium with eigenvalues given by $\lambda = (−1, 0\pm i)$ and a strange attractor
with Lyapunov exponents of (0.0394, 0, −1.4067). The equilibrium at the origin is
nonlinearly unstable, and the strange attractor is self-excited. Initial conditions close
to the attractor are (0, 1, 0.4), and the basin of attraction is relatively small.

\subsubsection{Equilibrium Type 10}
This system has a single equilibrium point with one real positive eigenvalue and a
complex conjugate pair with zero real parts. A system of this type is
\begin{equation}
\label{eq:anytype_equi_14}
    \begin{array}
      \dot{x} = y \\
      \dot{y} = -x+yz \\
      \dot{z} = 0.23z + y2 − 10xy.
    \end{array}
  \right.
\end{equation}
It has an equilibrium with eigenvalues given by $\lambda = (0.23, 0\pm i)$ and a strange attractor
with Lyapunov exponents of (0.1241, 0, −2.4424). The system is invariant under the
transformation $(x, y, z) \rightarrow (−x, −y, z)$, and the strange attractor is symmetric and
self-excited. Initial conditions close to the attractor are (0, 1, 1), and the basin of
attraction is very large.

\begin{figure}[htbp]
\centering
\includegraphics[width=0.6\textwidth]{chaos/anytype_04.png}
\caption{\label{fig:anytype_04}
Attractors for systems with a single hyperbolic equilibrium point for each of the
eight types projected onto the xz-plane. The equilibrium points are indicated by red dots
and lie in the y = 0 plane.
}
\end{figure}

\begin{figure}[htbp]
\centering
\includegraphics[width=0.6\textwidth]{chaos/anytype_05.png}
\caption{\label{fig:anytype_05}
Attractors for systems with a single nonhyperbolic equilibrium point for two of the
eleven types projected onto the xz-plane. The equilibrium points are indicated by red dots
and lie in the y = 0 plane.
}
\end{figure}

\subsection{Summary and conclusions}
Systems of autonomous ordinary differential equations of the form of Eq. (\ref{eq:anytype_equi_01}) admit
chaotic solutions with one or more strange attractors in the presence of a single
hyperbolic equilibrium point for each of the eight types as shown in Fig. \ref{fig:anytype_04} projected
onto the xz-plane. Two of the systems have a strange attractor coexisting with a
stable equilibrium, three of the systems have a symmetric pair of strange attractors,
and one (\ref{eq:anytype_06}) has two strange attractors and three coexisting limit cycles for the
given parameters. Five of the eight cases have hidden attractors.

There are eleven types of nonhyperbolic equilibrium points that can occur in threedimensional
systems. However, nine of the eleven types cannot occur in isolation for
the chosen system. The other two cases admit chaotic solutions with a single selfexcited
strange attractor as shown in Fig. \ref{fig:anytype_05}. All of the systems should have interesting
basins of attraction \cite{16Chaudhuri2014Complicated} and a rich set of bifurcations as the parameters are varied
and deserve further study.

Given that systems in the form of Eq. (\ref{eq:anytype_equi_01}) can exhibit chaos in the absence of
equilibrium points as in System \ref{eq:anytype_04}, it is not surprising that chaos can accompany
isolated equilibrium points of all the allowed types, although in most cases the equilibrium
point is relatively close to the attractor and thus would be expected to influence
its dynamics. The results presented here support the idea that any dynamic not explicitly
forbidden by some theorem will occur in an appropriately designed dynamical
system. One needs only to look carefully to find suitable examples. An interesting
question is whether strange attractors can occur in systems with the fourteen types
of hyperbolic equilibrium points that occur in four dimensions, but that answer will
be left to a subsequent publication.

\bibliographystyle{spphys}
\bibliography{chaos_any_equi}

