\chapter{Simple Chaotic Flows with A Line Equilibrium}
\label{chap:chaoslineequi}
\begin{pauthor}
  Xiong Wang\\
  Institute for Advanced Study,Shenzhen University,Nanshan District Shenzhen, Guangdong, China 518060\\
  \email{wangxiong8686@szu.edu.cn}\\
  Guanrong Chen\\
  Department of Electronic,Engineering, City University of Hong Kong, Hong Kong SAR, China.\\
  \email{eegchen@cityu.edu.hk}\\
  Sajad Jafari\\
  Biomedical Engineering Department, Amirkabir University of Technology, Tehran
  15875–4413, Iran\\
  \email{sajadjafari83@gmail.com}
\end{pauthor}

\section{Introduction}

Simple systems of nonlinear differential equations can exhibit chaos as we mentioned before.
And it's now possible to explore the entire parameter space of such systems with the development
of computer science.

Recent research on such systems has categorizing periodic and chaotic attractors as either self-excited
or hidden\cite{Kuznetsov2010Analytical,Kuznetsov2011Hidden,Leonov2011,
  Leonov20112494,Leonov20112230,Leonov2011Hidden,Leonov2012Hidden,Leonov2013Analytical,
  Leonov2012IWCFTA2012}. 
A self-excited attractor has a basin of attraction that is associated with an unstable equilibrium while
a hidden attractor has a basin of attraction that does not intersect with small neighborhoods
of any equilibrium points.
The classical attractor such as Lorenz, Rossler, Chua, Chen, Sprott systems (case B-S) and other
widely-know attractors are those excited from unstable equilibrium. From a computational point of
view, a trajectory starts from a point on the unstable manifold in the neighborhood of an unstable
equilibrium can reach an attractor and been identified\cite{Leonov20112230}. However, such method
is inapplicable for hidden attractor. Nevertheless, hidden attractors play important roles in
engineering that even a small perturbations in a structure like a bridge or an airplane wing may cause disasters.
The chaotic attractors in dynamical systems without any equilibrium points or with only
stable equilibrium are hidden attractors. That's the reason why such systems are rarely
found and only a few such examples have been reported\cite{Jafari2013Elementary,MALIHE2013SIMPLE,Wang2011A,Wang2012Constructing,Wei2012Dynamical,Wang2012A,Wei2011Dynamical,Wang2012AB,Wei2012Delayed,Wei2010Anti}.

In this chapter, we'll introduce chaotic systems with a line equilibrium of which attractors 
can categorised as hidden attractor.
Although in such systems, the basin of attraction may intersect with the line equilibrium  
in some section, it's impossible to identify the chaotic attractor by choosing an arbitrary initial 
condition in the vicinity of the unstable equilibrium due to uncountably many points on the line 
outside the basin of attraction of the chaotic attractor. From a computational point of view,
these attractors are hidden and knowledge about equilibrium does not help in their localization.

On the other hand, among those dynamical systems with a line equilibrium in the literature
\cite{MARCELO2012HOPF,Fiedler2000Generic,Fiedler2000Generic2,BERNOLD2012GENERIC},
only one chaotic example has been reported \cite{Zhou2013A} and is artificial for the reason that it's a four-dimensional
system that can be reduced to three-dimensional system in which the line equilibrium vanishes.

The goal of this chapter is to describe a new category of hidden attractor and expand the list of
known mathematically simple hidden chaotic attractors. We can perform a systematic computer
search for chaos in three-dimensional autonomous systems with quadratic nonlinearities which
have been designed so that there will be a line equilibrium and can not been made to vanish
by reduction to a system of lower dimension.

\begin{table}[htb]
\caption{\label{tab:line_01}Six simple chaotic flows with line equilibrium.}
\centering
\begin{tabular}{lllrrrrr}
\hline
\hline
Case & Equations & \((a,b)\) & Equilibrium & Eigenvalues & LEs & \(D_{KY}\) & \((x_0,y_0,z_0)\)\\
\hline
\(LE_1\) & \(\dot{x} = y\) & a = 15 & 0 & \(\frac{z\pm\sqrt{z^2-4}}{2}\) & 0.0717 & 2.1371 & 0\\
 & \(\dot{y} = -x + yz\) & b = 1 & 0 & 0 & 0 &  & 0.5\\
 & \(\dot{z} = -x - axy - bxz\) &  & z &  & -0.5232 &  & 0.5\\
\hline
\(LE_2\) & \(\dot{x} = y\) & a = 17 & 0 & \(\frac{z\pm\sqrt{z^2-4}}{2}\) & 0.0564 & 2.1927 & 0\\
 & \(\dot{y} = -x + yz\) & b = 1 & 0 & 0 & 0 &  & 0.4\\
 & \(\dot{z} = -y - axy - bxz\) &  & z &  & -0.2927 &  & 0\\
\hline
\(LE_3\) & \(\dot{x} = y\) & a = 18 & 0 & $\frac{z\pm{}\sqrt{z^2-4}}{2}$ & 0.0556 & 2.1714 & 0\\
 & \(\dot{y} = -x + yz\) & b = 1 & 0 & 0 & 0 &  & -0.4\\
 & \(\dot{z} = x^2 - axy - bxz\) &  & z &  & -0.3245 &  & 0.5\\
\hline
\(LE_4\) & \(\dot{x} = y\) & a = 4 & 0 & $\frac{z\pm{}\sqrt{z^2-4}}{2}$ & 0.0539 & 2.1712 & 0.2\\
 & \(\dot{y} = -x + yz\) & b = 0.6 & 0 & 0 & 0 &  & 0.7\\
 & \(\dot{z} = -axy - bxy - yz\) &  & z &  & -0.3147 &  & 0\\
\hline
\(LE_5\) & \(\dot{x} = y\) & a = 1.5 & 0 & $\frac{z\pm{}\sqrt{z^2-4a}}{2}$ & 0.1386 & 2.1007 & 0.7\\
 & \(\dot{y} = -ax + yz\) & b = 5 & 0 & 0 & 0 &  & 1\\
 & \(\dot{z} = -x^2 - y^2 - bxz\) &  & z &  & -1.3764 &  & 0\\
\hline
\(LE_6\) & \(\dot{x} = y\) & a = 0.04 & 0 & $\frac{z\pm{}\sqrt{z^2-4}}{2}$ & 0.0543 & 2.0860 & 12\\
 & \(\dot{y} = -x + yz\) & b = 0.1 & 0 & 0 & 0 &  & 2\\
 & \(\dot{z} = -ay^2 - xy - bxz\) &  & z &  & -0.6314 &  & 0\\
\hline
\(LE_7\) & \(\dot{x} = z\) & a = 1.85 & 0 & $\frac{-0.3y\pm{}\sqrt{0.09y^2-4y}}{2}$ & 0.1144 & 2.0140 & 5.1\\
 & \(\dot{y} = x + yz\) & b = 0.3 & y & 0 & 0 &  & 7\\
 & \(\dot{z} = -ax^2 - xy - byz\) &  & 0 &  & -1.0270 &  & 0\\
\hline
\(LE_8\) & \(\dot{x} = z\) & a = 3 & 0 & $\pm{}\sqrt{y}$ & 0.0521 & 2.0647 & 0\\
 & \(\dot{y} = -x - yz\) & b = 1 & y & 0 & 0 &  & -0.3\\
 & \(\dot{z} = ax^2 - xy - bxz\) &  & 0 &  & -0.8053 &  & -1\\
\hline
\(LE_9\) & \(\dot{x} = z\) & a = 1.62 & x & $\frac{0.62\pm{}\sqrt{6.8644-4x^2}}{2}$ & 0.0642 & 2.0939 & 0\\
 & \(\dot{y} = -ay + xz\) & b = 0.2 & 0 & 0 & 0 &  & 1\\
 & \(\dot{z} = z - bz^2 + xy\) &  & 0 &  & -0.6842 &  & 0.8\\
\hline
\hline
\end{tabular}
\end{table}

\section{Simple Chaotic Flows with A Line Equilibrium}

We start from a special case of Nose-Hoover oscillator \cite{Hoover1995Remark} which describes many
natural phenomena\cite{Posch1986Canonical}.
The form we'll focus on is the conservative Sprott case A system \cite{Sprott1994Some},

\begin{equation}
\label{eq:line_eq1}
  \left\{
    \begin{array}{l}
      \dot{x}=y\\
      \dot{y}=-x+yz \\
      \dot{z}=1-y^{2}.
    \end{array}
  \right.
\end{equation}

The system is the oldest and best-known example of a chaotic system with no equilibrium while
non-existence of a strange attractor in such system due to its conservation. We then consider
the parametric form of Eq. \ref{eq:line_eq1} with quadratic nonlinearities of the form,

\begin{equation}
\label{eq:line_eq2}
  \left\{
    \begin{array}{l}
      \dot{x}=y\\
      \dot{y}=a_1x+a_2yz \\
      \dot{z}=a_3x+a_4y+a_5x^{2}+a_6y^2+a_7xy+a_8xz+a_9yz.
    \end{array}
  \right.
\end{equation}

The system has a line equilibrium in $(0,0,z)$ with no other
equilibrium (i.e. the z-axis is the line equilibrium of such
system). With the consideration of millions of combinations
of coefficients $a_1$ through $a_9$ and initial conditions,
an exhaustive computer search has been applied in order to
discover the cases of which the largest Lyapunov exponent
is greater than 0.001. For each case been found, the space
of coefficients was searched for values that are deemed
``elegant'' \cite{Sprott2010Elegant}, by which we mean
that as many coefficients as possible been set to zero
while the rests are set to $\pm 1$ if possible or otherwise
to a small integer or decimal fraction with the fewest
possible digits.
In Table \ref{tab:line_01}, cases of $LE_1\sim{}LE_6$ are six
simple cases found with only six terms.
With some similar procedure, three other similar cases $LE_7–LE_9$
have been also found and listed in Table \ref{tab:line_01}.

There are dozens of additional cases have been found using
the similar procedure. Nevertheless, those are just equivalent to
cases in Table \ref{tab:line_01} with some linear transformation or
extensions of the cases in table with more than six terms.

As showed in Fig. \ref{fig:line_fig1}, attractors of these cases are
projected onto the xy-plane and the cases showed dissipation.
Table \ref{tab:line_01} lists the equilibrium, eigenvalues, Lyapunov
exponent spectra and Kaplan-Yorke dimensions along with initial
conditions that are close to the attractor.
Cases listed in Table \ref{tab:line_01} all have dimensions only
slightly greater than 2.0 which just as the usual cases 
for strange attractors from three-dimensional autonomous systems.
Among those listed in the table, the largest of which is $LE_2$ with
$D_{KY} = 2.1927$, and no effort has been made to tune the parameters
for maximum chaos. All the cases appear to approach chaos through a
succession of period-doubling limit cycles, a typical example
of which ($LE_1$) is shown in Fig. \ref{fig:line_fig2} with decreasing $a$ for
$b = 1$. As $a$ decreasing further, the strange attractor is destroyed
in a boundary crisis.

\begin{figure}[htbp]
\centering
\includegraphics[width=0.6\textwidth]{chaos/line_01.png}
\caption{\label{fig:line_fig1}
State space plots of the cases in Table \ref{tab:line_01} projected onto the xy-plane}
\end{figure}

\begin{figure}[htbp]
\centering
\includegraphics[width=0.6\textwidth]{chaos/line_02.png}
\caption{\label{fig:line_fig2}
  The largest Lyapunov exponent and bifurcation diagram of case $LE_1$ showing a period-doubling route to chaos}
\end{figure}

Fig. \ref{fig:line_fig3} shows a cross section in the xz-plane at $y = 0$ of
the basin of attraction for the two attractors for the typical
case $LE_1$. Note that the cross section of the strange attractor
nearly touches its basin boundary as is typical of lowdimensional
chaotic flows.

Eigenvalues of case $LE_1$ can be obtained from
$\lambda(\lambda^2-z\lambda+1)=0$. Despite the one
which equals to zero, the other eigenvalues depends
on the value of $z$. When $z>0$, using the Routh-Hurwitz
stability criterion, the two eigenvalues have positive parts
and thus the positive z-axis is unstable. As showed in Fig \ref{fig:line_fig3}
the basin of attraction of the chaotic attractor intersects
with the line equilibrium in some portions. Nevertheless, there are
other parts of the z-axis (for $z>0$ part) which lie in the basin
of the stable equilibrium or that attract to infinity. The strange
attractor is hidden in the sense that uncountable unstable points
on the line equilibrium of which only a tiny portion intersects
the basin of the chaotic attractor which means, from computational point of view,
the attractor is hidden and knowledge about the line equilibrium does not
help in its localization.

\begin{figure}[htbp]
\centering
\includegraphics[width=0.6\textwidth]{chaos/line_03.png}
\caption{\label{fig:line_fig3}
Cross section of the basins of attraction of the two attractors in the xz-plane at $y = 0$ for case $LE_1$. Initial conditions in the white region lead to
unbounded orbits, those in the red region lead to the strange attractor, and those in the light blue region lead to the line equilibrium.(For interpretation of
the references to color in this figure legend, the reader is referred to the web version of this article.)}
\end{figure}

The stable equilibrium for $z<-2$ does not appeared in the
basin plot for its node-like rather than a focus in the xy-plane.
The trajectory never crosses the $y=0$ plane for $z<-2$.
Orbits that start to the left of the equilibrium ($x < -2$) are
pulled in the $+z$ direction for $z > -2$ and in the $-z$ direction for $z < -2$.
However, not until starting far from the equilibrium, they do not go to infinity.
Rather, they asymptotically approach it from the $-\chi$ side and converge to a point
on the line that depends on the initial condition. Thus the entire negative
z-axis is an attractor, but it is nonlinearly contracting along
its length for $-2 < z < 0$ and nonlinearly expanding for $z < -2$.

Fig. \ref{line_fig4} shows the regions of different
dynamical behavior in the ab-parameter space, and as showed in figure, the behavior described above does not
depend on the particular choice of parameters.
Each pixel in the figure means a different initial condition from a Gaussian distribution
with zero mean and unit variance.
Thus the region in which the strange attractor (black dots) coexists with the
stable line equilibrium (light blue background) extends
throughout much of the parameter space. The other eight
cases in Table \ref{tab:line_01} shows similar behavior.
Also of interest is the fact that in cases $LE_1\sim{}LE_8$, the
strange attractor surrounds the line equilibrium, while in
case $LE_9$, the line equilibrium lies just outside the strange
attractor. In none of the cases does the line equilibrium
intersect the attractor, and thus we would not expect
homoclinic orbits.

\begin{figure}[htbp]
\centering
\includegraphics[width=0.6\textwidth]{chaos/line_04.png}
\caption{\label{fig:line_fig4}
Regions of different dynamic behavior in parameter space for case $LE_1$. Light blue represents a static equilibrium, and the black dots correspond to
regions of chaos. Each pixel uses a different random condition thereby indicating the coexistence of static and chaotic attractors. (For interpretation of the
references to color in this figure legend, the reader is referred to the web version of this article.)}
\end{figure}

\section{Conclusion}

In conclusion, it is apparent that simple chaotic systems
with a line equilibrium that seemed to be rare, may in fact
be rather common. These systems belong to the newly
introduced class of chaotic systems with hidden attractors.
In fact they are a new category of them which have not
been previously described.

\bibliographystyle{spphys}
\bibliography{chaos_line_equi}
