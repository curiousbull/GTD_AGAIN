% Created 2017-05-16 周二 21:53
% Intended LaTeX compiler: pdflatex
\documentclass[11pt]{ctexart}
                                      \usepackage[utf8]{inputenc}
                                      \usepackage[T1]{fontenc}
                                      \usepackage{fixltx2e}
                                      \usepackage{graphicx}
                                      \usepackage{longtable}
                                      \usepackage{float}
                                      \usepackage{wrapfig}
                                      \usepackage{rotating}
                                      \usepackage[normalem]{ulem}
                                      \usepackage{amsmath,bm,mathrsfs}
                                      \usepackage{textcomp}
                                      \usepackage{marvosym}
                                      \usepackage{wasysym}
                                      \usepackage{amssymb}
                                      \usepackage{booktabs}
                                      \usepackage[colorlinks,linkcolor=black,anchorcolor=black,citecolor=black]{hyperref}
                                      \tolerance=1000
                                      \usepackage{listings}
                                      \usepackage{xcolor}
                                      \lstset{
                                      %行号
                                      numbers=left,
                                      %背景框
                                      framexleftmargin=10mm,
                                      frame=none,
                                      %背景色
                                      %backgroundcolor=\color[rgb]{1,1,0.76},
                                      backgroundcolor=\color[RGB]{245,245,244},
                                      %样式
                                      keywordstyle=\bf\color{blue},
                                      identifierstyle=\bf,
                                      numberstyle=\color[RGB]{0,192,192},
                                      commentstyle=\it\color[RGB]{0,96,96},
                                      stringstyle=\rmfamily\slshape\color[RGB]{128,0,0},
                                      %显示空格
                                      showstringspaces=false
                                      }
\date{\today}
\title{}
\hypersetup{
 pdfauthor={},
 pdftitle={},
 pdfkeywords={},
 pdfsubject={},
 pdfcreator={Emacs 25.1.1 (Org mode 9.0.5)}, 
 pdflang={English}}
\begin{document}

\tableofcontents

\section{关于跨周期的处理}
\label{sec:orgcea9333}

\subsection{说明}
\label{sec:orgf13e268}

\begin{enumerate}
\item 研报中有一个概念:日线和小时线的择时效果不同,小时线反应更加迅速和准确,日线相对于小时线择时,
能够 \textbf{去除噪音的干扰} ,从而择时更加准确,而小时线的买卖点则更加及时。在原有不同的框架和周期
数据上, \textbf{切换择时信号和操作频率} 往往能够起到较好的择时效果。
\item CTA 领域中,也有类似理论,叫做 \emph{跨周期引用} ,一般是在额定交易周期下,引用一个更大周期(稳定)
数据,通过此数据做一些 \textbf{方向性保护} ,比如日线是多单状态,分钟线不要开空单。
\end{enumerate}

\subsection{实现方式}
\label{sec:org2bc8983}


\subsubsection{原理}
\label{sec:orgafa4ca1}

新编一个函数,用来根据小周期下的数据推算出大周期的 BAR 数据。返回的大周期数据将以序列变量的形式
保留在对应的小周期 K 线上,以便以后调用。

\subsubsection{函数}
\label{sec:org824b344}

\begin{enumerate}
\item 函数 \texttt{MtBar} 返回数值类型为数值类型

\begin{verbatim}
Params
        Numeric TimeFrame(1440);    
        // 目标时间周期:月线=40320,周线=10080,日线=1440,4 小时线=240
        // 其他 1 小时内的周期等于相应的分钟数,如:1 小时=60,30 分钟=30。。。
        // 支持不规则分钟数,如 3 分钟,8 分钟,之类都行

        Numeric BarsBack(1);
        // 目标时间周期 BAR 偏移:
        // 1--表示将目标时间周期下的前 1 根 K 线数据作为与当前 Bar 对应的目标时间周期下的 K 线数据
        // 0--表示将目标时间周期下的截止到目前为止的数据转换为与当前 BAR 对应的目标时间周期下 K 线数据

        NumericRef oCurBar;                 // 目标时间周期下的 Bar 索引
        NumericRef oOPenHT;         // 目标时间周期下的开盘价
        NumericRef oHighHT;         // 目标时间周期下的最高价
        NumericRef oLowHT;          // 目标时间周期下的最低价
        NumericRef oCloseHT;        // 目标时间周期下的收盘价
        NumericRef oVolHT;          // 目标时间周期下的成交量
        NumericRef oOpenIntHT;      // 目标时间周期下的持仓量

Vars
        NumericSeries barCnt;
        NumericSeries CurBar;
        NumericSeries barCntSum;
        NumericSeries OpenHT;
        NumericSeries HighHT;
        NumericSeries LowHT;
        NumericSeries CloseHT;
        NumericSeries VolHT;
        NumericSeries OpenIntHT;
        Numeric CurTime;
        Numeric PreTime;
        bool condition(false);
        Numeric i;
Begin
        If (TimeFrame == 40320)                 // 月线
        {
                CurTime = Month;
                PreTime = Month[1];
        }
        Else If (TimeFrame == 10080)                        // 周线
        {
                CurTime = IntPart(DateDiff(19700105,Date)/7);
                PreTime = IntPart(DateDiff(19700105,Date[1])/7);
        }
        Else                                                                        // 其他时间周期
        {
                CurTime = IntPart((DateDiff(19700105,date)*1440 + Hour*60 + Minute)/TimeFrame);
                PreTime = IntPart((DateDiff(19700105,date[1])*1440 + Hour[1]*60 + Minute[1])/TimeFrame);
        } 
        condition = CurTime != PreTime;

        If (CurrentBar==0)                // 如果是第一根 Bar, CurBar=0
        {
                barCnt = 0;
                CurBar = 0;
                OpenHT = Open;
                HighHT = High;
                LowHT = Low;
                CloseHT = Close;
                VolHT = Vol;
                OpenIntHT = OpenInt;
        }
        Else
        {
                If(Condition)                
                // 如果在目标周期下,属于另一根 K 线,则 CurBar 加 1
                {
                        barCnt = 1;
                        CurBar = CurBar[1] + 1;
                        OpenHT = Open;
                        HighHT = High;
                        LowHT = Low;
                        VolHT = Vol;
                }Else
                // 如果在目标周期下,属于同一根 K 线,则 CurBar 不变,但最高价和最低价要记录价格的变化,成交量要累加
                {
                        barCnt = barCnt[1] + 1;
                        CurBar = CurBar[1];
                        OpenHT = OpenHT[1];
                        HighHT = Max(HighHT[1],High);
                        LowHT = Min(LowHT[1],Low);
                        VolHT = VolHT[1] + Vol;
                }
                // 收盘价和持仓量总是取最新值
                CloseHT = Close;
                OpenIntHT = OpenInt;
        }

        // 上面的程序,在每根小周期的 K 线上,记录了它所属的大时间周期下的开高低收等值的变化。
        // 接下来,要把在大的时间周期级别上,属于同一根 K 线的开高低收这些数据,记录在这一组小周期 K 线的最后一根上。
        barCntSum = barCnt ;
        If(BarsBack == 0)
        // 如果 Bar 偏移参数为 0,则取每根小周期 K 线上保留的大时间周期截止到这根小周期 K 线为止的 BAR 数据
        {
                barCntSum = 0 ;
        }Else If(BarsBack == 1)
        // 如果 Bar 偏移参数为 1,则取大时间周期的上一根 K 线的 BAr 数据
        {
                barCntSum = barCnt ;
        }Else
        // 如果 BAR 偏移参数为其他,则取大时间周期的指定偏移后的那根 K 线的 BAR 数据
        {
                For i = 2 To BarsBack
                {
                        barCntSum = barCntSum + barCnt[barCntSum];
                }
        }

        // 最后将相应的 K 线数据作为引用参数返回
        oCurBar = CurBar;
        oOpenHT = OpenHT[barCntSum];
        oHighHT = HighHT[barCntSum];
        oLowHT = LowHT[barCntSum];
        oCloseHT = CloseHT[barCntSum];
        oVolHT = VolHT[barCntSum];
        oOpenIntHT = OpenIntHT[barCntSum];
        Return barCnt;
End
\end{verbatim}

\item 实现跨周期的求和函数 \texttt{MtSummation}

\begin{verbatim}
Params
        NumericSeries Price(1);
        NumericSeries BarCnt(0);
        Numeric Length(10);
Vars
        NumericSeries SumValue(0);
        Numeric i;
        Numeric j(0);
Begin
        SumValue = 0;
        For i = 1 to Length
        {
                If (Price[j] <> InvalidNumeric)
                {
                        SumValue = SumValue + Price[j];
                        j = j + BarCnt[j];
                }
                else Break;
        }
        Return SumValue;
End
\end{verbatim}

\item 实现计算跨周期简单移动平均的函数 \texttt{MtMa}

\begin{verbatim}
Params
        Numeric TimeFrame(1440);        // 目标时间周期参数,参数说明参见 MtBar
        Numeric BarsBack(1);                // 目标时间周期 BAR 偏移参数,说明见 MtBar 函数
        Numeric Length(10);                        // 均线周期
        NumericRef oMA;             // 以目标时间周期下的 K 线数据计算出的移动平均线
Vars
        NumericSeries mtBarCnt;
        NumericSeries mtClose;
        Numeric refCurBar;
        Numeric refOpen;
        Numeric refHigh;
        Numeric refLow;
        Numeric refClose;
        Numeric refVol;
        Numeric refOpenInt;

        Numeric SumValue(0);
        Numeric i;
        Numeric j(0);
Begin
        mtBarCnt = MtBar(TimeFrame,BarsBack,refCurBar,refOpen,refHigh,refLow,refClose,refVol,refOpenInt);
        mtClose = refClose;

        SumValue = MtSummation(mtClose,mtBarCnt,Length);
        oMA = SumValue/Length;
        Return mtBarCnt;
End
\end{verbatim}
\end{enumerate}
\subsubsection{应用}
\label{sec:org166d74b}
\begin{enumerate}
\item 以日线的均线交叉判断大趋势,然后在 5 分钟图上做交易
\item 日线的短期均线上穿长期均线,只做多,不做空;反之则只做空,不做多
\item 确定大趋势后,再根据 5 分钟图来判断小趋势,以决定进场时机。我们仍然用均线来判断,在多头趋势下,
如果 5 分钟的短期均线上穿长期均线,进场做多,反穿出场,但不反手做空;空头趋势下,类似。

\begin{verbatim}
Params
        Numeric TimeFrame(1440);        // 目标时间周期参数,参数说明参见 MtBar
        Numeric BarsBack(1);                // 目标时间周期 BAR 偏移参数,说明见 MtBar 函数

        Numeric Length1(10);                // 大周期的短期均线周期                
        Numeric Length2(20);                // 大周期的长期均线周期
        Numeric Length3(10);                // 小周期的短期均线周期
        Numeric Length4(20);                // 小周期的长期均线周期
        Numeric Lots(1);
Vars
        NumericSeries MA1;
        NumericSeries MA2;
        Numeric oMA1;
        Numeric oMA2;

        NumericSeries MA3;
        NumericSeries MA4;
Begin
        MtMa(TimeFrame,BarsBack,Length1,oMA1);
        MA1 = oMA1;
        PlotNumeric("MA1",MA1);
        MtMa(TimeFrame,BarsBack,Length2,oMA2);
        MA2 = oMA2;
        PlotNumeric("MA2",MA2);
        MA3 = AverageFC(Close,Length3);
        MA4 = AverageFC(Close,Length4);
        PlotNumeric("MA3",MA3);
        PlotNumeric("MA4",MA4);


        If (MA1>MA2)                // 大周期均线金叉,多头趋势
        {
                if (MarketPosition!=1 and MA3[1]>MA4[1])
                {
                        Buy(Lots,Open);
                }
                if (MarketPosition==1 and MA3[1]<MA4[1])
                {
                        Sell(Lots,Open);
                }
        }
        If (MA1<MA2)                // 大周期均线死叉,空头趋势
        {
                if (MarketPosition!=-1 and MA3[1]<MA4[1])
                {
                        SellShort(Lots,Open);
                }
                if (MarketPosition==-1 and MA3[1]>MA4[1])
                {
                        BuyToCover(Lots,Open);
                }
        }
End
\end{verbatim}
\end{enumerate}
\end{document}
